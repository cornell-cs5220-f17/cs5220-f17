\documentclass[12pt, leqno]{article} %% use to set typesize
\usepackage{fancyhdr}
\usepackage[sort&compress]{natbib}
\usepackage[letterpaper=true,colorlinks=true,linkcolor=black]{hyperref}

\usepackage{amsfonts}
\usepackage{amsmath}
\usepackage{amssymb}
\usepackage{color}
\usepackage{tikz}
\usepackage{pgfplots}
\usepackage{listings}
\usepackage{courier}
%\usepackage[utf8]{inputenc}
%\usepackage[russian]{babel}

\lstset{
  numbers=left,
  basicstyle=\ttfamily\footnotesize,
  numberstyle=\tiny\color{gray},
  stepnumber=1,
  numbersep=10pt,
}

\newcommand{\iu}{\ensuremath{\mathrm{i}}}
\newcommand{\bbR}{\mathbb{R}}
\newcommand{\bbC}{\mathbb{C}}
\newcommand{\calV}{\mathcal{V}}
\newcommand{\calW}{\mathcal{W}}
\newcommand{\macheps}{\epsilon_{\mathrm{mach}}}
\newcommand{\matlab}{\textsc{Matlab}}

\newcommand{\ddiag}{\operatorname{diag}}
\newcommand{\fl}{\operatorname{fl}}
\newcommand{\nnz}{\operatorname{nnz}}
\newcommand{\tr}{\operatorname{tr}}
\renewcommand{\vec}{\operatorname{vec}}

\newcommand{\vertiii}[1]{{\left\vert\kern-0.25ex\left\vert\kern-0.25ex\left\vert #1
    \right\vert\kern-0.25ex\right\vert\kern-0.25ex\right\vert}}
\newcommand{\ip}[2]{\langle #1, #2 \rangle}
\newcommand{\ipx}[2]{\left\langle #1, #2 \right\rangle}
\newcommand{\order}[1]{O( #1 )}

\newcommand{\kron}{\otimes}


\newcommand{\hdr}[2]{
  \pagestyle{fancy}
  \lhead{Bindel, Spring 2017}
  \rhead{Numerical Analysis (CS 4220)}
  \fancyfoot{}
  \begin{center}
    {\large{\bf #1}}
  \end{center}
  \lstset{language=matlab,columns=flexible}
}


\begin{document}
\hdr{2017-08-24}{Performance analysis}

Performance analysis is a rich area that combines experiment, theory,
and engineering. In this class, we will explore all three. The purpose
of this note is to set the stage; more specifically, we want to introduce

\begin{enumerate}
\item
  Performance analysis concepts that will recur
  throughout the course.
\item
  Common misconceptions and deceptions regarding
  performance
\end{enumerate}

High performance computing (HPC), like most engineering disciplines, is
about tradeoffs. The goal is to compute an answer that is good enough,
fast enough, within the constraints of the available computing
resources. But it is often not so simple to measure either solution
quality or resource constraints.

\section{What to measure?}

The usual measure of performance in HPC is \emph{time to solution}: the
number of seconds of \emph{wall clock time} to satisfy some termination
criterion. It is important to specify wall clock time rather than CPU
time, because fully utilizing all the CPUs in a large parallel machine
is extremely challenging! In a \emph{scaling study}, one attempts to
quantify how the time to solution depends on the computing resources
available, and possibly on the problem size or solution quality. To
understand the results of a scaling study, it is often useful to look at
measures derived from wall clock time, such as the effective flop rate,
speedup, scaled speedup, or parallel efficiency. In addition to wall
clock time, one might want to understand how much memory, disk space,
bandwidth, or power is used in a given computation. But figuring out
what to plot is a critical part of designing a performance study, and
more exotic measures (flops/GW?) sometimes do more to obscure
performance issues than to illuminate them.

\paragraph{Takeaway}

Think twice if you see anything but wall clock time to solution reported
as a primary performance measure!

\section{Basic performance concepts}

Most performance studies involve at least two parameters:

\begin{itemize}
\item
  $n$ -- a measure of the problem size
\item
  $p$ -- the number of processors
\end{itemize}

Understanding how run time varies these two parameters (and others)
gives a general framework for understanding performance. A study in
which $p$ varies is called a \emph{scaling study}, and is at the heart
of most parallel performance analyses.

\subsection{Algorithm complexity}

The starting point for most performance analysis involves understanding
how work scales with problem size. For example, computing the product of
two $n \times n$ floating point matrices takes about $2n^3$ flops
(floating point operations): $n^3$ adds and $n^3$ multiplies. In most
introductory computer science classes, we focus use order notation to
give a crude but concise description of how the work performed by an
algorithm scales with problem size; that is, matrix multiply is
$O(n^3)$. In an ideal world, we would say the time is proportional to
$n^3$, and determine the proportionality constant by a single
experiment. In practice, things are rarely so simple.

Simple asymptotic complexity models are absolutely the right place to
\emph{start} when thinking about work done by a computation, but they
are only a starting point. The ``hidden constant'' in the order notation
used in most complexity estimates can vary dramatically depending both
on the nature of the algorithm and on details of how it is implemented
-- a well-tuned matrix multiplication routine can be orders of magnitude
faster than a naive implementation on the same hardware. Moreover, in
modern machines, the cost to communicate data can easily exceed the cost
of floating point operations; an overly crude complexity estimate that
does not account for this fact may yield wildly inaccurate predictions.

\paragraph{Takeaway}

Algorithm complexity tells us how work scales with problem size --
useful, but not all there is to performance.

\subsection{Response time and throughput}

Some algorithms run to completion, then return a result more-or-less
instantaneously at the end. In many cases, though, a computation will
produce intermediate results. In this case, there are usually two
quantities of interest in characterizing the performance:

\begin{itemize}
\item
  The time to first result (response time)
\item
  Rate at which results are subsequently produced (throughput)
\end{itemize}

For problems involving information retrieval or communication, we
usually refer to these as the \emph{latency} and \emph{bandwidth}. But
the idea of an initial response time and subsequent throughput rate
applies more broadly.

When we think about concepts of latency and throughput, we are measuring
performance not by a single number (time to completion), but by a curve
of utility versus time. When we think about a \emph{fixed} latency and
throughput, we are implicitly defining a piecewise linear model of
utility versus time: there's an initial period of zero utility, followed
by a period of linearly increasing utility with constant slope (until
completion). The piecewise linear model is attractive in its simplicity,
but more complex models are sometimes useful. For example, to understand
the performance of an iterative solver, the right way to measure
performance might be in terms of approximation error versus time.

\paragraph{Takeaway}

Time to \emph{completion} is not always the right measure of progress.

\subsection{Theoretical machine limits}

The theoretical rate at which a code could run depends both on the
characteristics of the code and on characteristics of the machine on
which it runs. These include the \emph{theoretical peak flop rate} the
maximum rate at which floating point operations that could possibly be
executed. Even this is not so straightforward to estimate, since modern
processors are often capable of launching more than one floating point
instruction in each cycle, which may be vector floating point
instructions that can do several distinct floating point operations.

In addition to the computational capacity of the machine, it is
important to understand something about the \emph{latency} and
\emph{peak bandwidth} of any communication channels, I/O devices, and
memory subsystems. For I/O and memory, one may also have to worry about
the \emph{peak capacity}; even if there is ample storage for a
computation's data in main memory, it's hard to get good performance
unless the \emph{working set} that is most frequently accessed fits in
relatively fast (and small) cache memories.

We will discuss some basics of machine architecture in the coming
classes in order to set the context for understanding the theoretical
limits of the machine. For now, we simply note that these limits are not
so simple, and it is not simple to get to them: with tuning, a real code
might make it to 10\% of peak (depending on the nature of the
computation).

\subsection{Speedup, efficiency, and Amdahl's law}

In a \emph{strong scaling} study, one fixes the problem size $n$ and
studies performance as a function of $p$. The usual measure of
scalability is the \emph{speedup} of a parallel computation:
\[
  S(p) = \frac{T_{\mathrm{serial}}}{T_{\mathrm{parallel}}(p)}
\]
where $T_{\mathrm{serial}}$ is the performance of the \emph{best}
serial code available and $T_{\mathrm{parallel}}(p)$ is the time
required with $p$ processors. The \emph{parallel efficiency} is the
ratio of the speedup to the ideal linear speedup:
\[
  \mbox{Efficiency}(p)
    = \frac{S(p)}{p}
    = \frac{T_{\mathrm{serial}}}{p T_{\mathrm{parallel}}(p)}.
\]
A \emph{speedup plot} of speedup versus $p$ is a standard graphic in
most of the HPC literature, and is simultaneously one of the most useful
and one of the most abused plots around. We return to some of the issues
with deceptive speedup plots later in these notes.

We rarely achieve ideal linear speedup (100\% efficiency), in part
because most real codes include some work that is difficult or
impossible to parallelize. If $\alpha$ is the fraction of the serial
work that cannot be parallelized, then \emph{Amdahl's law} tells us the
best scaling we can hope for is
\[
  S(p) \leq \frac{1}{\alpha+(1-\alpha)/p} \leq \frac{1}{\alpha}.
\]
For example, if 10\% of the work in a given computation is serial, we
cannot hope for more than a $10 \times$ speedup no matter how many
processors we use. In practice, this is usually a generous estimate:
some overheads usually grow with the number of processors, so that past
a certain number of processors the speedup often doesn't just level off,
but actually \emph{decreases}.

\paragraph{Takeaway}

Speedup quantifies scalability. For fixed problems, Amdahl's law tells
us us that serial work limits the max speedup.

\subsection{Scaled speedup, weak scaling, and Gustafson's law}

In strong scaling studies, we assume we are interested in a fixed
problem size $n$. In many cases, though, we are not interested in using
ever-more parallelism to get faster answers to the same problems;
rather, we want to use increased parallelism to solve bigger problems.
In \emph{weak scaling} studies, we usually consider the \emph{scaled
speedup}
\[
S(p) = \frac{T_{\mathrm{serial}}(n(p))}
            {T_{\mathrm{parallel}}(n(p),p)}
\]
where $n(p)$ is a family of problem sizes chosen so that the work per
processor remains constant. For weak scaling studies, the analog of
Amdahl's law is \emph{Gustafson's law}; if $a$ is the amount of serial
work and $b$ is the parallelizable work, then
\[
  S(p) \leq \frac{a + bP}{a + b} = p-\alpha(p-1)
\]
where $\alpha = a/(a+b)$ is the fraction of serial work.

\paragraph{Takeaway}

Fixing the problem size (strong scaling) is not the only way. Sometimes
weak scaling studies are more instructive.

\subsection{Pleasing parallelism and high throughput}

A problem that can be decomposed into many independent tasks with little
overhead used to be called \emph{embarrassingly parallel}; these days,
it is sometimes called \emph{pleasingly parallel} instead. Monte Carlo
simulations and many ``big data analytics'' tasks are pleasingly
parallel. Because pleasingly parallel jobs have very low overheads
associated with serial work or with synchronization, they offer a high
level of scalability, even on machines where communication and
synchronization is expensive. Frameworks like Google's MapReduce thrive
in part because there are indeed many embarassingly parallel
computations in the world. At the same time, there are also many jobs
that are not embarrassingly parallel.

The communities that deal primarily with embarassingly parallel jobs
tend to be different than the traditional scientific HPC communities. In
particular, where HPC usually focuses on time to completion (with some
caveats noted above), communities focused on embarrassingly parallel
tasks usually care about high \emph{throughput}. If we sometimes refer
to HTC (high-throughput computing) in this class, this is what we mean.

\paragraph{Takeaway}

Some tasks are easy to parallelize. Some are not. It helps to know the
difference.

\subsection{Theoretical and empirical performance models}

\begin{quote}
With four parameters, I can fit an elephant, and with five, I can make
him wiggle his trunk.\\--
\href{https://en.wikiquote.org/wiki/John_von_Neumann}{Von Neumman}
\end{quote}

A \emph{performance model} predicts the performance of some code as a
function of problem size, parallelism, and perhaps other parameters.
Performance models are useful to the extent that they help us predict
whether we can meet a performance goal and to the extent that they can
guide us to where our codes most need improvement (or where they will
run into scaling bottlenecks on future machines). Because models reflect
our understanding, they may be incomplete; indeed, the most useful
models are \emph{necessarily} incomplete, since otherwise they are too
cumbersome to reason about! Experiments reflect what really happens, and
are a critical counterpoint to models.

The division between performance models and experiments is not sharp. In
the extreme case, machine learning and other empirical function fitting
methods can be used to estimate how performance depends on different
parameters under very weak assumptions. When strongly empirical models
have many parameters, a lot of data is needed to fit them well;
otherwise, the models may be \emph{overfit}, and do a poor job of
predicting performance except away from the training data. This may be
appropriate for cases where the model is used as the basis for
\emph{auto-tuning} a commonly-used kernel for a particular machine
architecture, for example. But performance experiments often aren't
cheap -- or at least they aren't cheap in the regime where people most
care about performance -- and so a simple, theory-grounded model is
often preferable. There's an art to balancing what should be modeled and
what should be treated semi-empirically, in performance analysis as in
the rest of science and engineering.

\paragraph{Takeaway}

Both theory and experiment are needed for performance modeling.

\subsection{Applications, benchmarks, and kernels}

The performance of application codes is usually what we really care
about. But application performance is generally complicated. The main
computation may involve alternating phases, each complex in its own
right, in addition to time to load data, initialize any data structures,
and post-process results. Because there are so many moving parts, its
also hard to use measurements of the end-to-end performance of a given
code on a given machine to infer anything about the speed expected of
other codes. Sometimes it's hard even to tell how the same code will run
on a different machine!

\emph{Benchmark codes} serve to provide a uniform way to compare the
performance of different machines on ``characteristic'' workloads.
Usually benchmark codes are simplified versions of real applications (or
of the computationally expensive parts of real applications); examples
include the \href{https://www.nas.nasa.gov/publications/npb.html}{NAS
parallel benchmarks}, the \href{http://www.graph500.org/}{Graph 500}
benchmarks, and (on a different tack) Sandia's
\href{https://mantevo.org/}{Mantevo} package of mini-applications.

\emph{Kernels} are frequently-used subroutines such as matrix multiply,
FFT, breadth-first search, etc. Because they are building blocks for so
many higher-level codes, we care about kernel performance a lot; and a
kernel typically involves a (relatively) simple computation. A common
first project in parallel computing classes is to time and tune a matrix
multiplication kernel.

Parallel \emph{patterns} (or
\href{www.eecs.berkeley.edu/Pubs/TechRpts/2006/EECS-2006-183.pdf}{``dwarfs''})
are abstract types of computation (like dense linear algebra or graph
analysis) that are higher level than kernels and more abstract than
benchmarks. Unlike a kernel or a benchmark, a pattern is too abstract to
benchmark. On the other hand, benchmarks can elucidate the performance
issues that occur on a given machine with a particular type of
computation.

\paragraph{Takeaway}

Application performance is complicated. We try to simplify by looking at
benchmark codes and kernels, or by understanding the performance
characteristics of common computational patterns.

\section{Designing performance experiments}

\subsection{Timing and profiling}

\emph{Profiling} involves running a code and measuring how much time
(and resources) are used in different parts of the code. For codes that
show any data-dependent performance, it is important to profile on
something realistic, as the time breakdown will depend on the use case.
One can profile with different levels of detail. The simplest case often
involves manually instrumenting a code with timers. There are also tools
that \emph{automatically instrument} either the source code or binary
objects to record timing information. \emph{Sampling profilers} work
differently; they use system facilities to interrupt the program
execution periodically and measure where the code is. It is also
possible to use \emph{hardware counters} to estimate the number of
performance-relevant events (such as cache misses or flops) that have
occurred in a given period of time. We will discuss these tools in more
detail as we get into the class (and we'll use some of them on our
codes).

As with everything else, there are tradeoffs in running a profiler:
methods that provide fine-grained information can produce a \emph{lot}
of data, enough that storing and processing profiling data can itself be
a challenge. There is also an issue that very fine-grained measurement
can interfere with the software being measured. It is often helpful to
start with a relatively crude, lightweight profiling technology in order
to first find what's interesting, then focus on the interesting bits for
more detailed experimentation.

Profiling has different goals. The most common reason to profile is to
find \emph{performance bottlenecks}: in many codes, the majority of the
time is spent in one or a few pieces of a large code base, and it makes
sense to find where the time is spent in order to spend tuning time in a
sensible way. Good compilers can also use profile data as the basis of
\emph{profile-directed optimization}.

\subsection{Experimental issues}

Compared to most experimental sciences, computer scientists have it
easy: our only safety issues involve RSI, and if an experiment breaks,
we can just re-run it with little additional work. But, as with other
experimental work, performance analysis does require that we understand
issues that lead to variation in results. For example:

\begin{itemize}
\item
  When running the same computation twice (in the same process), the
  second run will often be faster because the first run ``warms the
  cache.'' We will discuss this in more detail when we discuss computer
  architecture basics.
\item
  Two codes running on the same machine can interfere with each other
  (e.g.~by stealing memory bandwidth or thrashing shared caches), even
  if they are nominally not trying to use the same cores.
\item
  Timers have finite resolution, and so it may be necessary to run a
  short code segment repeatedly in order to get use enough time to get
  an accurate measurement.
\end{itemize}

All this suggests that it is important to understand the limitations of
experimental measurement, and also the environmental factors that cause
variations in measurements. We will return to these issues periodically
throughout the class.

\section{Engineering for performance}

Performance models and experiments help us both to design new codes for
high performance and to tune the performance of existing codes. We will
spend a lot of time talking about engineering for performance over the
course of the semester, but let's take a moment now to talk about a few
recurring themes.

\subsection{Know when to tune}

There are many reasons \emph{not} to tune code:

\begin{itemize}
\item
  Tuning takes human time. If the human time is more expensive than the
  computation time saved, it's not worth it.
\item
  Performance is often in tension with maintainability, generality, and
  other nice software design properties. If tuning for performance means
  making a mess of the code base, it may not be worth it.
\item
  Most codes have bottlenecks where the majority of the time is spent.
  It doesn't make sense to tune something that already takes little
  time.
\end{itemize}

We will mostly focus in the class on \emph{what} to tune, but in general
it is worth asking also \emph{why} a code should be tuned.

\subsection{Tune data structures}

On modern machines, memory access and communication patterns are
critical to performance. Because of this, tuning often involves looking
not at \emph{code} but at the \emph{data} that the code manipulates.
Many of the optimizations we will discuss in this class involve data
structures: rearranging arrays for unit stride, simplifying structures
with many levels of indirection, or using single precision for
high-volume floating point data, etc. With a proper interface
abstraction, the fastest way to high performance often involves
replacing a low-performance data structure with an equivalent
high-performance structure.

\subsection{Expose parallelism}

Achieving good performance on modern machines is increasingly about
making good use of parallel computing resources. This means not only
explicit parallelism (using threads or MPI, for example), but also
taking advantage of the parallelism available inside a single core. A
lot of the class will involve figuring out where there are opportunities
for parallelism, and exposing those opportunities to compilers,
frameworks, or ourselves!

\subsection{Use the right tools}

Performance tuning is hard. We make our lives easier by using the best
tools we can get our hands on: good compilers, well-tuned libraries and
frameworks, profilers and performance visualization tools, etc. And when
the right tools don't exist, sometimes we get to make them ourselves!

\section{Misconceptions and deceptions}

\begin{quote}
It ain't ignorance causes so much trouble; it's folks knowing so much
that ain't so.\\--
\href{http://www.famous-quotes.com/author.php?page=3\&total=81\&aid=733}{Josh
Billings}
\end{quote}

One of the common findings in pedagogy research is that an important
part of learning an area is overcoming common \emph{misconceptions}
about the area; see, e.g.
\href{http://www.lifescied.org/content/13/2/179.abstract}{{[}6{]}},
\href{http://aer.sagepub.com/content/50/5/1020}{{[}7{]}},
\href{http://sydney.edu.au/science/physics/pdfs/research/super/PhD(Muller).pdf}{{[}8{]}}.
And there are certainly some common misconceptions about high
performance computing! Some misconceptions are exacerbated by bad
reporting, which leads to deceptions and delusions about performance.

\subsection{Incorrect mental models}

\paragraph{Algorithm = implementation}

We can never time algorithms. We only time implementations, and
implementations vary in their performance. In some cases,
implementations may vary by orders of magnitude in their performance!

\paragraph{Asymptotic cost is always what matters}

We can't time algorithms, but we can reason about their asymptotic
complexity. When it comes to scaling for large $n$, the asymptotic
complexity can matter a lot. But comparing the asymptotic complexity of
two algorithms for modest $n$ often doesn't make sense! QuickSort may
not always be the fastest algorithm for sorting a list with ten
elements\ldots{}

\paragraph{Simple serial execution}

Hardware designers go to great length to present us with the
\emph{interface} that modern processor cores execute a instructions
sequentially. But this \emph{interface} is not the actual
\emph{implementation}. Behind the scenes, a simple stream of x86
instructions may be chopped up into micro-instructions, scheduled onto
different functional units acting in parallel, and executed out of
order. The \emph{effective behavior} is supposed to be consistent with
sequential execution -- at least, that's what happens on one core -- but
that illusion of sequential execution does not extend to performance.

\paragraph{Flops are all that count}

Data transfers from memory to the processor are often more expensive
than the computations that are run on that data.

\paragraph{Flop rates are what matter}

What matters is time to solution. Often, the algorithms that get the
best flop rates are not the most asymptotically efficient methods; as a
consequence, a code that uses the hardware less efficiently (in terms of
flop rate) may still give the quickest time to solution.

\paragraph{All speedup is linear}

See the comments above about Amdahl's law and Gustafson's law. We rarely
achieve linear speedup outside the world of embarrassingly parallel
applications.

\paragraph{All applications are equivalent}

Performance depends on the nature of the computation, the nature of the
implementation, and the nature of the hardware. Extrapolating
performance from one computational style, implementation, or hardware
platform to another is something that must be done very carefully.

\section{Deceptions and self-deceptions}

In 1991, David Bailey wrote an article on
\href{http://www.davidhbailey.com/dhbpapers/twelve-ways.pdf}{``Twelve
Ways to Fool the Masses When Giving Performance Results on Parallel
Computers''}. It's still worth reading (as are various follow-up pieces
-- see the Further Reading section), and highlights issues that we still
see now. To summarize slightly, here's my version of the list of common
performance deceptions:

\subsection{Unfair comparisons and strawmen}

A common sin in scaling studies is to compare the performance of a
parallel code on $p$ processors against the performance of the
\emph{same code} with $p = 1$. This ignores the fact that the parallel
code may have irrelevant overheads, or (worse) that there may be a
better organization for a single processor. Consequently, the speedups
no longer reflect the reasonable expectation of the reader that this is
the type of performance improvement they might see when going to a good
parallel implementation from a \emph{good} serial implementation. Of
course, it's also possible to see great speedups by comparing a bad
serial implementation to a correspondingly bad \emph{parallel}
implementation: a lot of unnecessary work can hide overheads.

A similar issue arises when computing with accelerators. Enthusiasts of
GPU-accelerated codes often claim order of magnitude (or greater)
performance improvements over using a CPU alone. Often, this comes from
explicitly tuning the GPU code and not the CPU code. A
\href{http://newport.eecs.uci.edu/~amowli/resources/papers/vuduc2010-hotpar.pdf}{2010
paper out of Georgia Tech} gives several examples where, after tuning,
two quad-core CPU sockets gave roughly the same performance as one or
two GPUs.

\subsection{Using the wrong measures}

If what you care about is time to solution, you might not really care so
much about watts per GFlop (though you certainly do if you're
responsible for supplying power for an HPC installation). More subtlely,
you don't necessarily care about scaled speedup if the natural problem
size is fixed (e.g.~in some graph processing applications).

\subsection{Deceptive plotting}

There are so many ways this can happen:

\begin{itemize}
\item
  Use of a log scale when one ought to have a linear scale, and
  vice-versa;
\item
  Choosing an inappropriately small range to exaggerate performance
  differences between near-equivalent options;
\item
  Not marking data points clearly, so that there is no visual difference
  between data falling on a straight line because it closely follows a
  trend and data falling on a straight line because there are two
  points.
\item
  Hiding poor scalability by plotting absolute time vs numbers of
  processors so that nobody can easily see that the time for 100
  processors (a small bar relative to the single-processor time) is
  equivalent to the time for 200 processors.
\item
  And more!
\end{itemize}

Plots allow readers to absorb trends very quickly, but it also makes it
easy to give wrong impressions.

\subsection{Too much faith in models}

Any model has limits of validity, and extrapolating outside those limits
leads to nonsense. Treat with due skepticism claims that -- according to
some model -- a code will run an order of magnitude faster in an
environment where it has not yet been run.

\subsection{Undisclosed tweaks}

There are many ways to improve performance. Sometimes, better hardware
does it; sometimes, better tuned code; sometimes, algorithmic
improvements. Claiming that a jump in performance comes from a new
algorithm without acknowledging differences in the level of tuning
effort, or acknowledging non-algorithmic changes (e.g.~moving from
double precision to single precision) is deceptive, but sadly common.
Hiding tweaks in the fine print in the hopes that the reader is skimming
doesn't make this any less deceptive!

\section{Rules for presenting performance results}

In the intro to the book
\href{http://www.lifescied.org/content/13/2/179.abstract}{Performance
Tuning of Scientific Applications}, David Bailey suggests nine
guidelines for presenting performance results without misleading the
reader. Paraphrasing only slightly, these are:

\begin{enumerate}
\item
  Follow rules on benchmarks
\item
  Only present actual performance, not extrapolations
\item
  Compare based on comparable levels of tuning
\item
  Compare wall clock times (not flop rates)
\item
  Compute performance rates from consistent operation counts based on
  the best serial codes.
\item
  Speedup should compare to best serial version. Scaled speedup plots
  should be clearly labeled and explained.
\item
  Fully disclose information affecting performance: 32/64 bit, use of
  assembly, timing of a subsystem rather than the full system, etc.
\item
  Don't deceive skimmers. Take care not to make graphics, figures, and
  abstracts misleading, even in isolation.
\item
  Report enough information to allow others to reproduce the results. If
  possible, this should include
\begin{itemize}
\item
  The hardware, software and system environment
\item
  The language, algorithms, data types, and coding techniques used
\item
  The nature and extent of tuning
\item
  The basis for timings, flop counts, and speedup computations
\end{itemize}
\end{enumerate}


\section{Questions}

\begin{enumerate}
\item
  A given program spends 10\% of its time in an initial startup phase,
  and then 90\% of its time in work that can be easily parallelized.
  Assuming a machine with homogeneous cores, plot the idealized speedup
  and parallel efficiency of the overall code according to Amdahl's law
  for up to 128 cores. If you know how, you should use a script to
  produce this plot, with both the serial fraction and the maximum
  number of cores as parameters.
\item
  Suppose a particular program can be partitioned into perfectly
  independent tasks, each of which takes time $\tau$. Tasks are set up,
  scheduled, and communicated to $p$ workers at a (serial) central
  server; this takes an overhead time $\alpha$ per task. What is the
  theoretically achievable throughput (tasks/time)?
\item
  Under what circumstances is it best to not tune?
\item
  The class cluster consists of eight nodes and fifteen Xeon Phi
  accelerator boards (details under the ``Computing platform'' section
  of \href{/syllabus.html}{the syllabus}). Based on an online search for
  information on these systems, what do you think is the theoretical
  peak flop rate (double-precision floating point operations per
  second)? Show how you computed this, and give URLs for where you got
  the parameters in your calculation. (We will return to this question
  again after we cover some computer architecture.)
\item
  What is the theoretical peak flop rate for your own machine?
\end{enumerate}

\section{Further reading}

\begin{enumerate}
\item
  David Bailey,
  \href{http://www.davidhbailey.com/dhbpapers/twelve-ways.pdf}{Twelve
  Ways to Fool the Masses When Giving Performance Results on Parallel
  Computers}.
\item
  David Bailey,
  \href{http://www.davidhbailey.com/dhbpapers/mislead.pdf}{Misleading
  Performance Reporting in the Supercomputing Field}.
\item
  David Bailey,
  \href{http://www.davidhbailey.com/dhbtalks/dhb-12ways.pdf}{Twelve Ways
  to Fool the Masses: Fast Forward to 2011}.
\item
  George Hager,
  \href{http://blogs.fau.de/hager/archives/category/fooling-the-masses}{Modern
  ``Fooling the Masses'' stunts blog posts}.
\item
  David Bailey, Robert Lucas, and Samuel Williams (eds),
  \href{http://www.amazon.com/Performance-Scientific-Applications-Chapman-Computational-ebook/dp/B00UV98OHG}{Performance
  Tuning of Scientific Applications}.
\item
  Richard Vuduc, Aparna Chandramowlishwaran, Jee Choi, Murat (Efe)
  Guney, and Aashay Shringarpure.
  \href{http://newport.eecs.uci.edu/~amowli/resources/papers/vuduc2010-hotpar.pdf}{On
  the Limits of GPU Acceleration}.
\end{enumerate}

\end{document}
