% FIXME: Swap out diagrams
\documentclass{beamer}

% Should be documentclass beamer

\mode<presentation>
{
%  \usetheme[hideothersubsections]{PaloAlto}
  \usetheme{metropolis}
  \setbeamercovered{transparent}
}

\usepackage{amsfonts}
\usepackage{amsmath}
\usepackage{amssymb}
\usepackage{color}
\usepackage{tikz}
\usepackage{pgfplots}
\usepackage{listings}
\usepackage{courier}
%\usepackage[utf8]{inputenc}
%\usepackage[russian]{babel}

\lstset{
  numbers=left,
  basicstyle=\ttfamily\footnotesize,
  numberstyle=\tiny\color{gray},
  stepnumber=1,
  numbersep=10pt,
}

\newcommand{\iu}{\ensuremath{\mathrm{i}}}
\newcommand{\bbR}{\mathbb{R}}
\newcommand{\bbC}{\mathbb{C}}
\newcommand{\calV}{\mathcal{V}}
\newcommand{\calW}{\mathcal{W}}
\newcommand{\macheps}{\epsilon_{\mathrm{mach}}}
\newcommand{\matlab}{\textsc{Matlab}}

\newcommand{\ddiag}{\operatorname{diag}}
\newcommand{\fl}{\operatorname{fl}}
\newcommand{\nnz}{\operatorname{nnz}}
\newcommand{\tr}{\operatorname{tr}}
\renewcommand{\vec}{\operatorname{vec}}

\newcommand{\vertiii}[1]{{\left\vert\kern-0.25ex\left\vert\kern-0.25ex\left\vert #1
    \right\vert\kern-0.25ex\right\vert\kern-0.25ex\right\vert}}
\newcommand{\ip}[2]{\langle #1, #2 \rangle}
\newcommand{\ipx}[2]{\left\langle #1, #2 \right\rangle}
\newcommand{\order}[1]{O( #1 )}

\newcommand{\kron}{\otimes}


\newcommand{\hdr}[2]{
  \title[CS 5220, Fall 2017]{CS 5220: #2}
  \author{David Bindel}
  \date{#1}
}


\hdr{2017-10-17}{Dense Linear Algebra}

\begin{document}

\begin{frame}
  \titlepage
\end{frame}


\begin{frame}
  \frametitle{Where we are}

  \begin{itemize}
  \item This week: {\em dense} linear algebra
  \item Next week: {\em sparse} linear algebra
  \end{itemize}
  
\end{frame}


\begin{frame}
  \frametitle{Numerical linear algebra in a nutshell}

  \begin{itemize}
  \item Basic problems
    \begin{itemize}
    \item Linear systems: $Ax = b$
    \item Least squares: minimize $\|Ax-b\|_2^2$
    \item Eigenvalues: $Ax = \lambda x$
    \end{itemize}
  \item Basic paradigm: matrix factorization
    \begin{itemize}
    \item $A = LU$, $A = LL^T$
    \item $A = QR$
    \item $A = V \Lambda V^{-1}$, $A = Q T Q^T$
    \item $A = U \Sigma V^T$
    \end{itemize}
  \item Factorization $\equiv$ switch to basis that makes problem easy
  \end{itemize}

\end{frame}

\begin{frame}
  \frametitle{Numerical linear algebra in a nutshell}

  Two flavors: dense and sparse
  \begin{itemize}
  \item Dense == common structures, no complicated indexing
    \begin{itemize}
    \item General dense (all entries nonzero)
    \item Banded (zero below/above some diagonal)
    \item Symmetric/Hermitian
    \item Standard, robust algorithms (LAPACK)
    \end{itemize}
  \item Sparse == stuff not stored in dense form!
    \begin{itemize}
    \item Maybe few nonzeros (e.g. compressed sparse row formats)
    \item May be implicit (e.g. via finite differencing)
    \item May be ``dense'', but with compact repn (e.g. via FFT)
    \item Most algorithms are iterative; wider variety, more subtle
    \item Build on dense ideas
    \end{itemize}
  \end{itemize}

\end{frame}


\begin{frame}
  \frametitle{History}

  BLAS 1 (1973--1977)
  \begin{itemize}
  \item Standard library of 15 ops (mostly) on vectors
    \begin{itemize}
    \item Up to four versions of each: S/D/C/Z
    \item Example: DAXPY
      \begin{itemize}
      \item Double precision (real)
      \item Computes $Ax+y$
      \end{itemize}
    \item Goals
      \begin{itemize}
      \item Raise level of programming abstraction
      \item Robust implementation (e.g. avoid over/underflow)
      \item Portable interface, efficient machine-specific implementation
      \end{itemize}
    \item BLAS 1 == $O(n^1)$ ops on $O(n^1)$ data
    \item Used in LINPACK (and EISPACK?)
    \end{itemize}
  \end{itemize}
  
\end{frame}


\begin{frame}
  \frametitle{History}
  
  BLAS 2 (1984--1986)
  \begin{itemize}
  \item Standard library of 25 ops (mostly) on matrix/vector pairs
    \begin{itemize}
    \item Different data types and matrix types
    \item Example: DGEMV
      \begin{itemize}
      \item Double precision
      \item GEneral matrix
      \item Matrix-Vector product
      \end{itemize}
    \end{itemize}
  \item Goals
    \begin{itemize}
    \item BLAS1 insufficient
    \item BLAS2 for better vectorization (when vector machines roamed)
    \end{itemize}
  \item BLAS2 == $O(n^2)$ ops on $O(n^2)$ data
  \end{itemize}
\end{frame}


\begin{frame}
  \frametitle{History}

  BLAS 3 (1987--1988)
  \begin{itemize}
  \item Standard library of 9 ops (mostly) on matrix/matrix
    \begin{itemize}
    \item Different data types and matrix types
    \item Example: DGEMM
      \begin{itemize}
      \item Double precision
      \item GEneral matrix
      \item Matrix-Matrix product
      \end{itemize}
    \item BLAS3 == $O(n^3)$ ops on $O(n^2)$ data
    \end{itemize}
  \item Goals
    \begin{itemize}
    \item Efficient cache utilization!
    \end{itemize}
  \end{itemize}
\end{frame}


\begin{frame}
  \frametitle{BLAS goes on}
  
  \begin{itemize}
  \item \url{http://www.netlib.org/blas}
  \item CBLAS interface standardized
  \item Lots of implementations (MKL, Veclib, ATLAS, Goto, ...)
  \item Still new developments (XBLAS, tuning for GPUs, ...)
  \end{itemize}

\end{frame}


\begin{frame}
  \frametitle{Why BLAS?}

  Consider Gaussian elimination.

  \vspace{5mm}
  LU for $2 \times 2$:
  \[
  \begin{bmatrix}
    a & b \\
    c & d
  \end{bmatrix} =
  \begin{bmatrix}
    1 & 0 \\
    c/a & 1
  \end{bmatrix}
  \begin{bmatrix}
    a & b \\
    0 & d-bc/a
  \end{bmatrix}
  \]
  Block elimination
  \[
  \begin{bmatrix}
    A & B \\
    C & D
  \end{bmatrix} =
  \begin{bmatrix}
    I & 0 \\
    CA^{-1} & I
  \end{bmatrix}
  \begin{bmatrix}
    A & B \\
    0 & D-CA^{-1}B
  \end{bmatrix} 
  \]

  \vspace{5mm}
  Block LU
  \[
  \begin{bmatrix}
    A & B \\
    C & D
  \end{bmatrix} = 
  \begin{bmatrix}
    L_{11} & 0 \\
    L_{12} & L_{22} 
  \end{bmatrix}
  \begin{bmatrix}
    U_{11} & U_{12} \\
    0 & U_{22} 
  \end{bmatrix} =
  \begin{bmatrix}
    L_{11} U_{11} & L_{11} U_{12} \\
    L_{12} U_{11} & L_{21} U_{12} + L_{22} U_{22}
  \end{bmatrix}
  \]

\end{frame}


\begin{frame}[fragile]
  \frametitle{Why BLAS?}

  Block LU
  \[
  \begin{bmatrix}
    A & B \\
    C & D
  \end{bmatrix} = 
  \begin{bmatrix}
    L_{11} & 0 \\
    L_{12} & L_{22} 
  \end{bmatrix}
  \begin{bmatrix}
    U_{11} & U_{12} \\
    0 & U_{22} 
  \end{bmatrix} =
  \begin{bmatrix}
    L_{11} U_{11} & L_{11} U_{12} \\
    L_{12} U_{11} & L_{21} U_{12} + L_{22} U_{22}
  \end{bmatrix}
  \]

Think of $A$ as $k \times k$, $k$ moderate:
\begin{lstlisting}
[L11,U11] = small_lu(A);   % Small block LU
U12 = L11\B;               % Triangular solve
L12 = C/U11;               % "
S   = D-L21*U12;           % Rank k update
[L22,U22] = lu(S);         % Finish factoring
\end{lstlisting}
Three level-3 BLAS calls!
\begin{itemize}
\item Two triangular solves
\item One rank-$k$ update
\end{itemize}

\end{frame}


\begin{frame}
  \frametitle{LAPACK}

  LAPACK (1989--present): \url{http://www.netlib.org/lapack}
  \begin{itemize}
  \item Supercedes earlier LINPACK and EISPACK
  \item High performance through BLAS
    \begin{itemize}
    \item Parallel to the extent BLAS are parallel (on SMP)
    \item Linear systems and least squares are nearly 100\% BLAS 3
    \item Eigenproblems, SVD --- only about 50\% BLAS 3
    \end{itemize}
  \item Careful error bounds on everything
  \item Lots of variants for different structures
  \end{itemize}
    
\end{frame}


\begin{frame}
  \frametitle{ScaLAPACK}

  ScaLAPACK (1995--present): \url{http://www.netlib.org/scalapack}
  \begin{itemize}
  \item MPI implementations
  \item Only a small subset of LAPACK functionality
  \end{itemize}
\end{frame}


\begin{frame}
  \frametitle{PLASMA and MAGMA}

  PLASMA and MAGMA (2008--present):
  \begin{itemize}
  \item Parallel LA Software for Multicore Architectures
    \begin{itemize}
    \item Target: Shared memory multiprocessors
    \item Stacks on LAPACK/BLAS interfaces
    \item Tile algorithms, tile data layout, dynamic scheduling
    \item Other algorithmic ideas, too (randomization, etc)
    \end{itemize}
  \item Matrix Algebra for GPU and Multicore Architectures
    \begin{itemize}
    \item Target: CUDA, OpenCL, Xeon Phi
    \item Still stacks (e.g. on CUDA BLAS)
    \item Again: tile algorithms + data, dynamic scheduling
    \item Mixed precision algorithms (+ iterative refinement)
    \end{itemize}
  \item Dist memory: PaRSEC / DPLASMA
  \end{itemize}
\end{frame}

\end{document}

%% \begin{frame}
%%   \frametitle{Why is ScaLAPACK not all of LAPACK?}

%%   Consider what LAPACK contains...
%% \end{frame}


%% \begin{frame}
%%   \frametitle{Decoding LAPACK names}

%%   \begin{itemize}
%%   \item F77 $\implies$ limited characters per name
%%   \item General scheme:
%%     \begin{itemize}
%%     \item Data type (double/single/double complex/single complex)
%%     \item Matrix type (general/symmetric, banded/not banded)
%%     \item Operation type
%%     \end{itemize}
%%   \item Example: DGETRF
%%     \begin{itemize}
%%     \item Double precision
%%     \item GEneral matrix
%%     \item TRiangular Factorization
%%     \end{itemize}
%%   \item Example: DSYEVX
%%     \begin{itemize}
%%     \item Double precision
%%     \item General SYmmetric matrix
%%     \item EigenValue computation, eXpert driver
%%     \end{itemize}
%%   \end{itemize}
%% \end{frame}


%% \begin{frame}
%%   \frametitle{Structures}
  
%%   \begin{itemize}
%%   \item General: general (GE), banded (GB), pair (GG), tridiag (GT)
%%   \item Symmetric: general (SY), banded (SB), packed (SP), tridiag (ST)
%%   \item Hermitian: general (HE), banded (HB), packed (HP)
%%   \item Positive definite (PO), packed (PP), tridiagonal (PT)
%%   \item Orthogonal (OR), orthogonal packed (OP)
%%   \item Unitary (UN), unitary packed (UP)
%%   \item Hessenberg (HS), Hessenberg pair (HG)
%%   \item Triangular (TR), packed (TP), banded (TB), pair (TG)
%%   \item Bidiagonal (BD)
%%   \end{itemize}

%% \end{frame}


%% \begin{frame}
%%   \frametitle{LAPACK routine types}

%%   \begin{itemize}
%%   \item Linear systems (general, symmetric, SPD)
%%   \item Least squares (overdetermined, underdetermined, constrained, weighted)
%%   \item Symmetric eigenvalues and vectors
%%     \begin{itemize}
%%     \item Standard: $Ax = \lambda x$
%%     \item Generalized: $Ax = \lambda B x$
%%     \end{itemize}
%%   \item Nonsymmetric eigenproblems
%%     \begin{itemize}
%%     \item Schur form: $A = Q T Q^T$
%%     \item Eigenvalues/vectors
%%     \item Invariant subspaces
%%     \item Generalized variants
%%     \end{itemize}
%%   \item SVD (standard/generalized)
%%   \item Different interfaces
%%     \begin{itemize}
%%     \item Simple drivers
%%     \item Expert drivers with error bounds, extra precision, etc
%%     \item Low-level routines
%%     \item ... and ongoing discussions! (e.g. about C interfaces)
%%     \end{itemize}
%%   \end{itemize}
%% \end{frame}

\begin{frame}[fragile]
  \frametitle{Matrix vector product}

  Simple $y = Ax$ involves two indices
  \[
    y_i = \sum_{j} A_{ij} x_j
  \]
  Can organize around either one:
\begin{lstlisting}
% Row-oriented
for i = 1:n
  y(i) = A(i,:)*x;
end

% Col-oriented
y = 0;
for j = 1:n
  y = y + A(:,j)*x(j);
end
\end{lstlisting}
... or deal with index space in other ways!

\end{frame}

\begin{frame}
  \frametitle{Parallel matvec: 1D row-blocked}

  \begin{center}
    \includegraphics[width=0.5\textwidth]{figs/lec12mv1.pdf}
  \end{center}

  Receive broadcast ${\color{red} x_0}, {\color{blue} x_1}, {\color{green} x_2}$
  into local $x_0$, $x_1$, $x_2$; then
  \begin{align*}
    \mbox{On {\color{red}P0:}\quad}
    {\color{red} A_{00}} x_0 + 
    {\color{red} A_{01}} x_1 +
    {\color{red} A_{02}} x_2 &= {\color{red} y_0} \\
    \mbox{On {\color{blue}P1:}\quad}
    {\color{blue} A_{10}} x_0 +
    {\color{blue} A_{11}} x_1 +
    {\color{blue} A_{12}} x_2 &= {\color{blue} y_1} \\
    \mbox{On {\color{green}P2:}\quad}
    {\color{green} A_{20}} x_0 +
    {\color{green} A_{21}} x_1 +
    {\color{green} A_{22}} x_2 &= {\color{green} y_2}
  \end{align*}

\end{frame}

\begin{frame}[fragile]
  \frametitle{Parallel matvec: 1D col-blocked}

  \begin{center}
    \includegraphics[width=0.5\textwidth]{figs/lec12mv2.pdf}
  \end{center}

  Independently compute \\
  \begin{minipage}{0.3\textwidth}
    \[
      {\color{red}
        z^{(0)} =
        \begin{bmatrix} A_{00} \\ A_{10} \\ A_{20} \end{bmatrix} x_0
      }
      \]
  \end{minipage}
  \begin{minipage}{0.3\textwidth}
    \[
      {\color{blue}
        z^{(1)} = 
        \begin{bmatrix} A_{00} \\ A_{10} \\ A_{20} \end{bmatrix} x_1
      }
      \]
  \end{minipage}
  \begin{minipage}{0.3\textwidth}
    \[
      {\color{green}
        z^{(2)} = 
        \begin{bmatrix} A_{00} \\ A_{10} \\ A_{20} \end{bmatrix} x_2
      }
      \]
  \end{minipage}

  \vspace{3mm}
  and perform reduction: 
  $y = {\color{red} z^{(0)}} + {\color{blue} z^{(1)}} + {\color{green} z^{(2)}}$.

\end{frame}

\begin{frame}
  \frametitle{Parallel matvec: 2D blocked}
  
  \begin{center}
    \includegraphics[width=0.5\textwidth]{figs/lec12mv3.pdf}
  \end{center}

  \begin{itemize}
  \item Involves broadcast {\em and} reduction
  \item ... but with subsets of processors
  \end{itemize}
\end{frame}

\begin{frame}[fragile]
  \frametitle{Parallel matvec: 2D blocked}

  Broadcast ${\color{red} x_0}, {\color{yellow} x_1}$ to
  local copies $x_0, x_1$ at {\color{red} P0} and {\color{blue} P2} \\
  Broadcast ${\color{red} x_2}, {\color{yellow} x_3}$ to
  local copies $x_2, x_3$ at {\color{yellow} P1} and {\color{green} P3} \\

  \vspace{1mm}
  In parallel, compute
    \begin{align*}
      {\color{red}
      \begin{bmatrix}
        A_{00} & A_{01} \\
        A_{10} & A_{11}
      \end{bmatrix}}
      \begin{bmatrix}
        x_0 \\ x_1
      \end{bmatrix} &=
      {\color{red}
      \begin{bmatrix}
        z_0^{(0)} \\ z_1^{(0)}
      \end{bmatrix}}
&
      {\color{yellow}
      \begin{bmatrix}
        A_{02} & A_{03} \\
        A_{12} & A_{13}
      \end{bmatrix}}
      \begin{bmatrix}
        x_2 \\ x_3
      \end{bmatrix} &=
      {\color{yellow}
      \begin{bmatrix}
        z_0^{(1)} \\ z_1^{(1)}
      \end{bmatrix}}
\\
      {\color{blue}
      \begin{bmatrix}
        A_{20} & A_{21} \\
        A_{30} & A_{31}
      \end{bmatrix}}
      \begin{bmatrix}
        x_0 \\ x_1
      \end{bmatrix} &=
      {\color{blue}
      \begin{bmatrix}
        z_2^{(3)} \\ z_3^{(3)}
      \end{bmatrix}}
&
      {\color{green}
      \begin{bmatrix}
        A_{20} & A_{21} \\
        A_{30} & A_{31}
      \end{bmatrix}}
      \begin{bmatrix}
        x_0 \\ x_1
      \end{bmatrix} &=
      {\color{green}
      \begin{bmatrix}
        z_2^{(3)} \\ z_3^{(3)}
      \end{bmatrix}}
      \end{align*}

\vspace{1mm}
Reduce across rows:
\begin{align*}
  \begin{bmatrix}
    {\color{red} y_0} \\
    {\color{yellow} y_1}
  \end{bmatrix} &=
  {\color{red}
  \begin{bmatrix}
    z_0^{(0)} \\
    z_1^{(0)}
  \end{bmatrix}} +
  {\color{yellow}
  \begin{bmatrix}
    z_0^{(1)} \\
    z_1^{(1)}
  \end{bmatrix}}
&
  \begin{bmatrix}
    {\color{blue} y_2} \\
    {\color{green} y_3}
  \end{bmatrix} &=
  {\color{blue}
  \begin{bmatrix}
    z_2^{(2)} \\
    z_3^{(2)}
  \end{bmatrix}} +
  {\color{green}
  \begin{bmatrix}
    z_2^{(3)} \\
    z_3^{(3)}
  \end{bmatrix}}
\end{align*}
\end{frame}


\begin{frame}
  \frametitle{Parallel matmul}

  \begin{itemize}
  \item Basic operation: $C = C+AB$
  \item Computation: $2n^3$ flops
  \item Goal: $2n^3/p$ flops per processor, minimal communication
  \end{itemize}
\end{frame}


\begin{frame}
  \frametitle{1D layout}
  
  \begin{center}
    \includegraphics[width=\textwidth]{figs/lec12mm1.pdf}
  \end{center}

  \begin{itemize}
  \item Block MATLAB notation: $A(:,j)$ means $j$th block column
  \item Processor $j$ owns $A(:,j)$, $B(:,j)$, $C(:,j)$
  \item $C(:,j)$ depends on {\em all} of $A$, but only $B(:,j)$
  \item How do we communicate pieces of $A$?
  \end{itemize}

\end{frame}


\begin{frame}
  \frametitle{1D layout on bus (no broadcast)}

  \begin{center}
    \includegraphics[width=0.8\textwidth]{figs/lec12mm1.pdf}
  \end{center}

  \begin{itemize}
  \item Everyone computes local contributions first
  \item {\color{red} P0} sends {\color{red} $A(:,0)$} to
    each processor $j$ in turn; \\
    processor $j$ receives, computes $A(:,0) B(0,j)$
  \item {\color{blue} P1} sends {\color{blue} $A(:,1)$} to
    each processor $j$ in turn; \\
    processor $j$ receives, computes $A(:,1) B(1,j)$
  \item {\color{green} P2} sends {\color{green} $A(:,2)$} to
    each processor $j$ in turn; \\
    processor $j$ receives, computes $A(:,2) B(2,j)$
  \end{itemize}
\end{frame}


\begin{frame}
  \frametitle{1D layout on bus (no broadcast)}

  \begin{center}
    \includegraphics[width=\textwidth]{figs/lec12mm2.pdf}
  \end{center}
\end{frame}


\begin{frame}[fragile]
  \frametitle{1D layout on bus (no broadcast)}

\begin{lstlisting}
C(:,myproc) += A(:,myproc)*B(myproc,myproc)
for i = 0:p-1
  for j = 0:p-1
    if (i == j)      continue;
    if (myproc == i) i
      send A(:,i) to processor j
    if (myproc == j)
      receive A(:,i) from i
      C(:,myproc) += A(:,i)*B(i,myproc)
    end
  end
end
\end{lstlisting}
Performance model?
\end{frame}
 

\begin{frame}
  \frametitle{1D layout on bus (no broadcast)}

  No overlapping communications, so in a simple $\alpha-\beta$ model:
  \begin{itemize}
  \item $p(p-1)$ messages
  \item Each message involves $n^2/p$ data
  \item Communication cost: $p(p-1) \alpha + (p-1) n^2 \beta$
  \end{itemize}
\end{frame}



\begin{frame}
  \frametitle{1D layout on ring}

  \begin{center}
    \includegraphics[width=0.8\textwidth]{figs/lec12mm3.pdf}
  \end{center}

  \begin{itemize}
  \item Every process $j$ can send data to $j+1$ simultaneously
  \item Pass slices of $A$ around the ring until everyone sees
    the whole matrix ($p-1$ phases).
  \end{itemize}
\end{frame}


\begin{frame}[fragile]
  \frametitle{1D layout on ring}

\begin{lstlisting}
tmp = A(myproc)
C(myproc) += tmp*B(myproc,myproc)
for j = 1 to p-1
  sendrecv tmp to myproc+1 mod p, 
           from myproc-1 mod p
  C(myproc) += tmp*B(myproc-j mod p, myproc)
\end{lstlisting}
Performance model?
\end{frame}
 

\begin{frame}
  \frametitle{1D layout on ring}

  In a simple $\alpha-\beta$ model, at each processor:
  \begin{itemize}
  \item $p-1$ message sends (and simultaneous receives)
  \item Each message involves $n^2/p$ data
  \item Communication cost: $(p-1) \alpha + (1-1/p) n^2 \beta$
  \end{itemize}
\end{frame}

\begin{frame}[fragile]
  \frametitle{Outer product algorithm}

  Serial: Recall outer product organization:
\begin{lstlisting}
for k = 0:s-1
  C += A(:,k)*B(k,:);
end
\end{lstlisting}

  \vspace{5mm}
  Parallel: Assume $p = s^2$ processors, block $s \times s$ matrices. \\
  For a $2 \times 2$ example:
  \[
  \begin{bmatrix}
    C_{00} & C_{01} \\
    C_{10} & C_{11}
  \end{bmatrix} =
  \begin{bmatrix}
    A_{00} B_{00} & A_{00} B_{01} \\
    A_{10} B_{00} & A_{10} B_{01} 
  \end{bmatrix} +
  \begin{bmatrix}
    A_{01} B_{10} & A_{01} B_{11} \\
    A_{11} B_{10} & A_{11} B_{11} 
  \end{bmatrix}
  \]

\begin{itemize}
\item
  Processor for each $(i,j)$ $\implies$ parallel work for each $k$!
\item
  Note everyone in row $i$ uses $A(i,k)$ at once, \\
  and everyone in row $j$ uses $B(k,j)$ at once.
\end{itemize}

\end{frame}


\begin{frame}[fragile]
  \frametitle{Parallel outer product (SUMMA)}

\begin{lstlisting}
for k = 0:s-1
  for each i in parallel
    broadcast A(i,k) to row
  for each j in parallel
    broadcast A(k,j) to col
  On processor (i,j), C(i,j) += A(i,k)*B(k,j);
end
\end{lstlisting}
If we have tree along each row/column, then
\begin{itemize}
\item $\log(s)$ messages per broadcast
\item $\alpha + \beta n^2/s^2$ per message
\item $2 \log(s) (\alpha s + \beta n^2/s)$ total communication 
\item Compare to 1D ring: $(p-1) \alpha + (1-1/p) n^2 \beta$
\end{itemize}

\vspace{2mm}
Note: Same ideas work with block size $b < n/s$

\end{frame}


\begin{frame}
  \frametitle{Cannon's algorithm}

  \[
  \begin{bmatrix}
    C_{00} & C_{01} \\
    C_{10} & C_{11}
  \end{bmatrix} =
  \begin{bmatrix}
    A_{00} B_{00} & A_{01} B_{11} \\
    A_{11} B_{10} & A_{10} B_{01} 
  \end{bmatrix} +
  \begin{bmatrix}
    A_{01} B_{10} & A_{00} B_{01} \\
    A_{10} B_{00} & A_{11} B_{11} 
  \end{bmatrix}
  \]

  \vspace{5mm}
  Idea: Reindex products in block matrix multiply 
  \begin{align*}
    C(i,j) &= \sum_{k = 0}^{p-1} A(i,k) B(k,j) \\
          &= \sum_{k = 0}^{p-1} A(i,\, k+i+j \mod p) \; B(k+i+j \mod p, j)
  \end{align*}
  For a fixed $k$, a given block of $A$ (or $B$) is needed for
  contribution to {\em exactly one} $C(i,j)$.

\end{frame}


\begin{frame}[fragile]
  \frametitle{Cannon's algorithm}

\begin{lstlisting}
% Move A(i,j) to A(i,i+j) 
for i = 0 to s-1
  cycle A(i,:) left by i

% Move B(i,j) to B(i+j,j)
for j = 0 to s-1
  cycle B(:,j) up by j

for k = 0 to s-1
  in parallel;
    C(i,j) = C(i,j) + A(i,j)*B(i,j);
  cycle A(:,i) left by 1
  cycle B(:,j) up by 1
\end{lstlisting}

\end{frame}


\begin{frame}
  \frametitle{Cost of Cannon}

  \begin{itemize}
  \item Assume 2D torus topology
  \item Initial cyclic shifts: $\leq s$ messages each ($\leq 2s$ total)
  \item For each phase: $2$ messages each ($2s$ total)
  \item Each message is size $n^2/s^2$
  \item Communication cost: 
    $4s(\alpha + \beta n^2/s^2) = 4(\alpha s + \beta n^2/s)$
  \item This communication cost is optimal! \\
    ... but SUMMA is simpler, more flexible, almost as good
  \end{itemize}
\end{frame}


\begin{frame}[fragile]
  \frametitle{Speedup and efficiency}

  Recall
  \begin{align*}
    \mathrm{Speedup}    & := t_{\mathrm{serial}} / t_{\mathrm{parallel}} \\
    \mathrm{Efficiency} & := \mathrm{Speedup}/p
  \end{align*}
  
  Assuming no overlap of communication and computation, efficiencies are
  \begin{center}
  \begin{tabular}{ll}
    1D layout & $\left( 1+O\left( \frac{p}{n} \right) \right)^{-1}$ \\
    SUMMA &  $\left( 1+O\left( \frac{\sqrt{p} \log p}{n} \right) \right)^{-1}$ \\
    Cannon & $\left (1+O\left( \frac{\sqrt{p}}{n} \right) \right)^{-1}$
  \end{tabular}
  \end{center}
\end{frame}



\end{document}
