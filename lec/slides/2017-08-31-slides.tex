\documentclass{beamer}

% Should be documentclass beamer

\mode<presentation>
{
%  \usetheme[hideothersubsections]{PaloAlto}
  \usetheme{metropolis}
  \setbeamercovered{transparent}
}

\usepackage{amsfonts}
\usepackage{amsmath}
\usepackage{amssymb}
\usepackage{color}
\usepackage{tikz}
\usepackage{pgfplots}
\usepackage{listings}
\usepackage{courier}
%\usepackage[utf8]{inputenc}
%\usepackage[russian]{babel}

\lstset{
  numbers=left,
  basicstyle=\ttfamily\footnotesize,
  numberstyle=\tiny\color{gray},
  stepnumber=1,
  numbersep=10pt,
}

\newcommand{\iu}{\ensuremath{\mathrm{i}}}
\newcommand{\bbR}{\mathbb{R}}
\newcommand{\bbC}{\mathbb{C}}
\newcommand{\calV}{\mathcal{V}}
\newcommand{\calW}{\mathcal{W}}
\newcommand{\macheps}{\epsilon_{\mathrm{mach}}}
\newcommand{\matlab}{\textsc{Matlab}}

\newcommand{\ddiag}{\operatorname{diag}}
\newcommand{\fl}{\operatorname{fl}}
\newcommand{\nnz}{\operatorname{nnz}}
\newcommand{\tr}{\operatorname{tr}}
\renewcommand{\vec}{\operatorname{vec}}

\newcommand{\vertiii}[1]{{\left\vert\kern-0.25ex\left\vert\kern-0.25ex\left\vert #1
    \right\vert\kern-0.25ex\right\vert\kern-0.25ex\right\vert}}
\newcommand{\ip}[2]{\langle #1, #2 \rangle}
\newcommand{\ipx}[2]{\left\langle #1, #2 \right\rangle}
\newcommand{\order}[1]{O( #1 )}

\newcommand{\kron}{\otimes}


\newcommand{\hdr}[2]{
  \title[CS 5220, Fall 2017]{CS 5220: #2}
  \author{David Bindel}
  \date{#1}
}

\hdr{2017-08-31}{Optimization basics}

\begin{document}


\begin{frame}
  \titlepage
\end{frame}


\begin{frame}
  \frametitle{Reminder: Modern processors}

  \begin{itemize}
  \item Modern CPUs are
    \begin{itemize}
    \item Wide: start / retire multiple instructions per cycle
    \item Pipelined: overlap instruction executions
    \item Out-of-order: dynamically schedule instructions
    \end{itemize}
  \item Lots of opportunities for instruction-level parallelism (ILP)
  \item Complicated!  Want the compiler to handle the details
  \item Implication: we should give the compiler
    \begin{itemize}
    \item Good instruction mixes
    \item Independent operations
    \item Vectorizable operations
    \end{itemize}
  \end{itemize}
  
\end{frame}


\begin{frame}
  \frametitle{Reminder: Memory systems}

  \begin{itemize}
  \item Memory access are expensive!
  \item Flop time $\ll$ bandwidth$^{-1}$ $\ll$ latency
  \item Caches provide intermediate cost/capacity points
  \item Cache benefits from
    \begin{itemize}
    \item Spatial locality (regular local access)
    \item Temporal locality (small working sets)
    \end{itemize}
  \end{itemize}
  
\end{frame}


\begin{frame}
  \frametitle{Goal: (Trans)portable performance}

  \begin{itemize}
  \item Attention to detail has orders-of-magnitude impact
  \item Different systems = different micro-architectures, caches
  \item Want (trans)portable performance across HW
  \item Need {\em principles} for high-perf code along with
    tricks
  \end{itemize}
\end{frame}


\begin{frame}
  \frametitle{Basic principles}

  \begin{itemize}
  \item Think before you write
  \item Time before you tune
  \item Stand on the shoulders of giants
  \item Help your tools help you
  \item Tune your data structures
  \end{itemize}
  
\end{frame}


% =====================================================


\begin{frame}
  \frametitle{Think before you write}
\end{frame}


\begin{frame}
  \frametitle{Premature optimization}

  \begin{quote}
  We should forget about small efficiencies, say about 97\% of the
  time: premature optimization is the root of all evil. \\
  \hfill -- Don Knuth
  \end{quote}
\end{frame}


\begin{frame}
  \frametitle{Premature optimization}

  Wrong reading: ``Performance doesn't matter''
  \begin{quote}
  We should forget about small efficiencies, say about 97\% of the
  time: premature {\bf optimization is the root of all evil}. \\
  \hfill -- Don Knuth
  \end{quote}
\end{frame}


\begin{frame}
  \frametitle{Premature optimization}

  What he actually said (with my emphasis)
  \begin{quote}
  We should forget about {\bf small} efficiencies, say {\bf about 97\%} of the
  time: {\bf premature} optimization is the root of all evil. \\
  \hfill -- Don Knuth
  \end{quote}

  \begin{itemize}
  \item Don't forget the big efficiencies!
  \item Don't forget the 3\%!
  \item Your code is not premature forever!
  \end{itemize}
\end{frame}


\begin{frame}
  \frametitle{Don't sweat the small stuff}

  \begin{itemize}
  \item Speed-up from tuning $\epsilon$ of code $< (1-\epsilon)^{-1}
    \approx 1 + \epsilon$
  \item OK to write high-level stuff in Matlab or Python
  \item OK if configuration file reader is un-tuned
  \item OK if $O(n^2)$ prelude to $O(n^3)$ algorithm is not hyper-tuned?
  \end{itemize}
\end{frame}


\begin{frame}[fragile]
  \frametitle{Lay-of-the-land thinking}

  \begin{lstlisting}
    for (i = 0; i < n; ++i)
      for (j = 0; j < n; ++j)
        for (k = 0; k < n; ++k)
          C[i+j*n] += A[i+k*n] * B[i+k*n];
  \end{lstlisting}
  \begin{itemize}
  \item What are the ``big computations'' in my code?
  \item What are the natural algorithmic variants?
    \begin{itemize}
    \item Vary loop orders?  Different interpretations!
    \item Lower complexity algorithm (Strassen?)
    \end{itemize}
  \item Should I rule out some options in advance?
  \item How can I code so it is easy to experiment?
  \end{itemize}
\end{frame}


\begin{frame}
  \frametitle{How big is $n$?}

  Typical analysis: time is $O(f(n))$
  \begin{itemize}
  \item Meaning: $\exists C, N : \forall n \geq N, T_n \leq C f(n)$.
  \item Says {\em nothing} about constant factors: $O(10 n) = O(n)$
  \item Ignores lower order term: $O(n^3 + 1000 n^2) = O(n^3)$
  \item Behavior at small $n$ may not match behavior at large $n$!
  \end{itemize}
  Beware asymptotic complexity arguments about small-$n$ codes!
  
\end{frame}


\begin{frame}[fragile]
  \frametitle{Avoid work}

\begin{lstlisting}
bool any_negative1(int* x, int n)
{
  bool result = false;
  for (int i = 0; i < n; ++i)
    result = (result || x[i] < 0);
  return result;
}

bool any_negative2(int* x, int n)
{
  for (int i = 0; i < n; ++i)
    if (x[i] < 0)
      return false;
  return true;
}
\end{lstlisting}
\end{frame}


\begin{frame}
  \frametitle{Be cheap}

  Fast enough, right enough $\implies$ \\
  \hfill Approximate when you can get away with it.
\end{frame}


\begin{frame}
  \frametitle{Do more with less (data)}

  Want lots of work relative to data loads:
  \begin{itemize}
  \item Keep data compact to fit in cache
  \item Use short data types for better vectorization
  \item But be aware of tradeoffs!
    \begin{itemize}
    \item For integers: may want 64-bit ints sometimes!
    \item For floating-point: will discuss in detail in other lectures
    \end{itemize}
  \end{itemize}
\end{frame}


\begin{frame}
  \frametitle{Remember the I/O!}

  Example: Explicit PDE time stepper on $256^2$ mesh
  \begin{itemize}
  \item 0.25 MB per frame (three fit in L3 cache)
  \item Constant work per element (a few flops)
  \item Time to write to disk $\approx$ 5 ms
  \end{itemize}
  If I write once every 100 frames, how much time is I/O?
\end{frame}


% =====================================================


\begin{frame}
  \frametitle{Time before you tune}
\end{frame}


\begin{frame}
  \frametitle{Hot spots and bottlenecks}

  \begin{itemize}
  \item Often a little bit of code takes most of the time
  \item Usually called a ``hot spot'' or bottleneck
  \item Goal: Find and eliminate
    \begin{itemize}
    \item Cute coinage: ``de-slugging''
    \end{itemize}
  \end{itemize}
  
\end{frame}


\begin{frame}
  \frametitle{Practical timing}

  Need to worry about:
  \begin{itemize}
  \item System timer resolutions
  \item Wall-clock time vs CPU time
  \item Size of data collected vs how informative it is
  \item Cross-interference with other tasks
  \item Cache warm-start on repeated timings
  \item Overlooked issues from too-small timings
  \end{itemize}
\end{frame}


\begin{frame}
  \frametitle{Manual instrumentation}

  Basic picture:
  \begin{itemize}
  \item Identify stretch of code to be timed
  \item Run it several times with ``characteristic'' data
  \item Accumulate the total time spent
  \end{itemize}
  Caveats: Effects from repetition, ``characteristic'' data
\end{frame}


\begin{frame}
  \frametitle{Manual instrumentation}

  \begin{itemize}
  \item Hard to get {\em portable} high-resolution wall-clock time!
  \item Solution: {\tt omp\_get\_wtime()}
  \item Requires OpenMP support (still not CLang)
  \end{itemize}
\end{frame}


\begin{frame}
  \frametitle{Types of profiling tools}

  \begin{itemize}
  \item Sampling vs instrumenting
    \begin{itemize}
    \item Sampling: Interrupt every $t_{\mathrm{profile}}$ cycles
    \item Instrumenting: Rewrite code to insert timers
    \item Instrument at binary or source level
    \end{itemize}
  \item Function level or line-by-line
    \begin{itemize}
    \item Function: Inlining can cause mis-attribution
    \item Line-by-line: Usually requires debugging symbols ({\tt -g})
    \end{itemize}
  \item Context information?
    \begin{itemize}
    \item Distinguish full call stack or not?
    \end{itemize}
  \item Time full run, or just part?
  \end{itemize}
\end{frame}


\begin{frame}
  \frametitle{Hardware counters}

  \begin{itemize}
  \item Counters track cache misses, instruction counts, etc
  \item Present on most modern chips
  \item May require significant permissions to access...
  \end{itemize}
\end{frame}


\begin{frame}
  \frametitle{Automated analysis tools}

  \begin{itemize}
  \item Examples: MAQAO and IACA
  \item Symbolic execution of {\em model} of a code segment
  \item Usually only practical for short segments
  \item But can give detailed feedback on (assembly) quality
  \end{itemize}
\end{frame}


% =====================================================


\begin{frame}
  \frametitle{Shoulders of giants}
\end{frame}


\begin{frame}
  \frametitle{What makes a good kernel?}

  Computational kernels are
  \begin{itemize}
  \item Small and simple to describe
  \item General building blocks (amortize tuning work)
  \item Ideally high arithmetic intensity
    \begin{itemize}
    \item Arithmetic intensity = flops/byte
    \item Amortizes memory costs
    \end{itemize}
  \end{itemize}
  
\end{frame}


\begin{frame}
  \frametitle{Case study: BLAS}

  Basic Linear Algebra Subroutines
  \begin{itemize}
  \item Level 1: $O(n)$ work on $O(n)$ data
  \item Level 2: $O(n^2)$ work on $O(n^2)$ data
  \item Level 3: $O(n^3)$ work on $O(n^2)$ data
  \end{itemize}
  Level 3 BLAS are key for high-perf transportable LA.
\end{frame}


\begin{frame}
  \frametitle{Other common kernels}

  \begin{itemize}
  \item Apply sparse matrix (or sparse matrix powers)
  \item Compute an FFT
  \item Sort a list
  \end{itemize}
\end{frame}


\begin{frame}
  \frametitle{Kernel trade-offs}

  \begin{itemize}
  \item Critical to get {\em properly tuned} kernels
    \begin{itemize}
    \item Kernel {\em interface} is consistent across HW types
    \item Kernel {\em implementation} varies according to arch details
    \end{itemize}
  \item General kernels {\em may} leave performance on the table
    \begin{itemize}
    \item Ex: General matrix-matrix multiply for structured matrices
    \end{itemize}
  \item Overheads may be an issue for small $n$ cases
    \begin{itemize}
    \item Ex: Usefulness of batched BLAS extensions
    \end{itemize}
  \item But: Ideally, someone else writes the kernel!
    \begin{itemize}
    \item Or it may be automatically tuned
    \end{itemize}
  \end{itemize}
\end{frame}


% =====================================================


\begin{frame}
  \frametitle{Help your tools help you}
\end{frame}


\begin{frame}
  \frametitle{What can your compiler do for you?}

  In decreasing order of effectiveness:
  \begin{itemize}
  \item Local optimization
    \begin{itemize}
    \item Especially restricted to a ``basic block''
    \item More generally, in ``simple'' functions
    \end{itemize}
  \item Loop optimizations
  \item Global (cross-function) optimizations
  \end{itemize}
\end{frame}


\begin{frame}
  \frametitle{Local optimizations}

  \begin{itemize}
  \item Register allocation: compiler > human
  \item Instruction scheduling: compiler > human
  \item Branch joins and jump elim: compiler > human?
  \item Constant folding and propogation: humans OK
  \item Common subexpression elimination: humans OK
  \item Algebraic reductions: humans definitely help
  \end{itemize}
\end{frame}


\begin{frame}
  \frametitle{Loop optimizations}

  {\em Mostly} leave these to modern compilers
  \begin{itemize}
  \item Loop invariant code motion
  \item Loop unrolling
  \item Loop fusion
  \item Software pipelining
  \item Vectorization
  \item Induction variable substitution
  \end{itemize}

\end{frame}


\begin{frame}
  \frametitle{Obstacles for the compiler}

  \begin{itemize}
  \item Long dependency chains
  \item Excessive branching
  \item Pointer aliasing
  \item Complex loop logic
  \item Cross-module optimization
  \item Function pointers and virtual functions
  \item Unexpected FP costs
  \item Missed algebraic reductions
  \item Lack of instruction diversity
  \end{itemize}
  Let's look at a few...
\end{frame}


\begin{frame}[fragile]
  \frametitle{Ex: Long dependency chains}

  Sometimes these can be decoupled (e.g.~reduction loops)
  \begin{lstlisting}
  // Version 0
  float s = 0;
  for (int i = 0; i < n; ++i)
    s += x[i];
  \end{lstlisting}
  Apparent linear dependency chain.  Compilers might handle this, but
  let's try ourselves...
\end{frame}


\begin{frame}[fragile]
  \frametitle{Ex: Long dependency chains}

  Key: Break up chains to expose parallel opportunities
  \begin{lstlisting}
  // Version 1
  float s[4] = {0, 0, 0, 0};
  int i;

  // Sum start of list in four independent sub-sums
  for (i = 0; i < n-3; i += 4)
    for (int j = 0; j < 4; ++j)
    s[j] += x[i+j];

  // Combine sub-sums and handle trailing elements
  float s = (s[0]+s[1]) + (s[2]+s[3]);
  for (; i < n; ++i)
    s += x[i];
\end{lstlisting}  
\end{frame}


\begin{frame}[fragile]
  \frametitle{Ex: Pointer aliasing}

Why can this not vectorize easily?
\begin{lstlisting}
void add_vecs(int n, double* result, double* a, double* b)
{
  for (int i = 0; i < n; ++i)
    result[i] = a[i] + b[i];
}
\end{lstlisting}
Q: What if {\tt result} overlaps {\tt a} or {\tt b}?
\end{frame}


\begin{frame}[fragile]
  \frametitle{Ex: Pointer aliasing}

  C99: Use {\tt restrict} keyword
  \begin{lstlisting}
  void add_vecs(int n, double* restrict result,
                double* restrict a, double* restrict b);
  \end{lstlisting}
  Implicit promise: these point to different things in memory.

  \vspace{5mm}
  Fortran forbids aliasing --- part of why naive Fortran speed beats
  naive C speed!
\end{frame}

% List of obstacles and some solutions


\begin{frame}[fragile]
  \frametitle{Ex: ``Black box'' function calls}

Compiler must assume arbitrary wackiness from ``black box'' function calls
\begin{lstlisting}
double foo(double* restrict x)
{
  double y = *x;  // Load x once
  bar();    // Assume bar is a 'black box' fn
  y += *x;  // Must reload x
  return y;
}
\end{lstlisting}

\end{frame}


\begin{frame}
  \frametitle{Ex: Floating point issues}

  Several possible optimizations available:
  \begin{itemize}
  \item Use different precisions
  \item Use more/less accurate special function routines
  \item Underflow is flush-to-zero or gradual
  \end{itemize}
  Problem: This changes semantics!
  \begin{itemize}
  \item A daring compiler will pretend floats are reals and hope
  \item This will break some of my codes!
  \item Human intervention is indicated
  \end{itemize}
\end{frame}


\begin{frame}
  \frametitle{Optimization flags}

  \begin{itemize}
  \item {\tt -O[0123]} (no optimization -- aggressive optimization)
    \begin{itemize}
    \item {\tt -O2} is usually the default
    \item {\tt -O3} is useful, but might break FP codes (for example)
    \end{itemize}
  \item Architecture targets
    \begin{itemize}
    \item Usually a ``native'' mode targets current architecture
    \item Not always the right choice (e.g.~consider Totient head/compute)
    \end{itemize}
  \item Specialized optimization flags
    \begin{itemize}
    \item Turn on/off specific optimization features
    \item Often the basic {\tt -Ox} has reasonable defaults
    \end{itemize}
  \end{itemize}
  
\end{frame}


\begin{frame}
  \frametitle{Auto-vectorization and compiler reports}

  \begin{itemize}
  \item Good compilers try to vectorize for you
    \begin{itemize}
    \item Intel is pretty good at this
    \item GCC / CLang are OK, not as strong
    \end{itemize}
  \item Can get reports about what prevents vectorization
    \begin{itemize}
    \item Not necessarily by default!
    \item Helps a lot for tuning
    \end{itemize}
  \end{itemize}
\end{frame}


\begin{frame}
  \frametitle{Profile-guided optimization}

  Basic workflow:
  \begin{itemize}
  \item Compile code with optimizations
  \item Run in a profiler
  \item Compile again, provide profiler results
  \end{itemize}
  Helps compiler optimize branches based on observations.
\end{frame}


% =====================================================


\begin{frame}
  \frametitle{Data layout matters}
\end{frame}


\begin{frame}
  \frametitle{``Speed-of-light'' analysis}

  For compulsory misses to load cache:
  \[
  T_{\mbox{data}} \mbox{ (s)}
  \quad \geq \quad
  \frac{\mbox{data required (bytes)}}
       {\mbox{peak BW (bytes/s)}}
  \]
  Possible optimizations:
  \begin{itemize}
  \item Shrink working sets to fit in cache (pay this once)
  \item Use simple unit-stride access patterns
  \end{itemize}
  Reality is generally more complicated...  
\end{frame}


\begin{frame}[fragile]
  \frametitle{When and how to allocate}

Why is this an $O(n^2)$ loop?
\begin{lstlisting}[language=Matlab]
  x = [];
  for i = 1:n
    x(i) = i;
  end
\end{lstlisting}  
\end{frame}


\begin{frame}
  \frametitle{When and how to allocate}

  \begin{itemize}
  \item Access is not the only cost!
    \begin{itemize}
    \item Allocation / de-allocation also costs something
    \item So does garbage collection (where supported)
    \item Beware hidden allocation costs (e.g. on resize)
    \item Often bites naive library users
    \end{itemize}
  \item Two thoughts to consider
    \begin{itemize}
    \item Pre-allocation (avoid repeated alloc/free)
    \item Lazy allocation (if alloc will often not be needed)
    \end{itemize}
  \end{itemize}
\end{frame}


\begin{frame}
  \frametitle{Storage layout}

  Desiderata:
  \begin{itemize}
  \item Compact (fit lots into cache)
  \item Traverse with simple access patterns
  \item Avoids pointer chasing
  \end{itemize}
\end{frame}


\begin{frame}
  \frametitle{Multi-dimensional arrays}

  Two standard formats:
  \begin{itemize}
  \item Col-major (Fortran): Each column stored consecutively
  \item Row-major (C/C++): Each row stored consecutively
  \end{itemize}
  Ideally, traverse arrays with unit stride!  Layout affects choice.

  \vspace{1cm}
  More sophisticated multi-dim array layouts may be useful...
\end{frame}


\begin{frame}
  \frametitle{Blocking / tiling}

  Classic example: Matrix multiply
  \begin{itemize}
  \item Load $b \times b$ block of $A$
  \item Load $b \times b$ block of $B$
  \item Compute product of blocks
  \item Accumulate into $b \times b$ block of $C$
  \end{itemize}
  Have $O(b^3)$ work for $O(b^2)$ memory references!
\end{frame}


\begin{frame}
  \frametitle{Data alignment and vectorization}

  \begin{itemize}
  \item Vector load/stores faster if {\em aligned} (start at memory
    addresses that are multiples of 64 or 256)
  \item Can ask for aligned blocks of memory from allocator
  \item Then want aligned offsets into aligned blocks
  \item Have to help compiler recognize aligned pointers!
  \end{itemize}
\end{frame}


\begin{frame}
  \frametitle{Data alignment and cache contention}

  Issue: What if strided access causes conflict misses?
  \begin{itemize}
  \item Example: Walk across row of col-major matrix
  \item Example: Parallel arrays of large-power-of-2 size
  \end{itemize}
  Not the most common problem, but one to watch for.
  
\end{frame}


\begin{frame}[fragile]
  \frametitle{Structure layouts}

  \begin{itemize}
  \item Want $b$-byte type to start on $b$-byte memory boundary.
  \item Compiler may pad structures to enforce this.
  \item Advice: arrange structure fields in decreasing size order.
  \end{itemize}
\end{frame}


\begin{frame}[fragile]
  \frametitle{SoA vs AoS}

\begin{lstlisting}
// Struct of Arrays (parallel arrays)
typedef struct {
  double* x;
  double* y;
} aos_points_t;

// Array of Structs
typedef struct {
  double x;
  double y;
} point_t;
typedef point_t* soa_points_t;
\end{lstlisting}

\end{frame}


\begin{frame}
  \frametitle{SoA vs AoS}

  \begin{itemize}
  \item SoA: Structure of Arrays
    \begin{itemize}
    \item Friendly to vectorization
    \item Poor locality to access all of one item
    \item Awkward for lots of libraries and programs
    \end{itemize}
  \item AoS: Array of Structs
    \begin{itemize}
    \item Naturally supported default
    \item Not very SIMD-friendly
    \end{itemize}
  \item Possible to combine the two...
  \end{itemize}
\end{frame}


\begin{frame}
  \frametitle{Copy optimizations}

  Copy between formats to accelerate computations, e.g.
  \begin{itemize}
  \item Copy piece of AoS to SoA format
  \item Perform vector operations on SoA data
  \item Copy back out
  \end{itemize}
  Performance gains > copy costs?  Plays great with tiling!
  
\end{frame}


\begin{frame}
  \frametitle{For the control freak}

  Can get (some) programmer control over
  \begin{itemize}
  \item Pre-fetching
  \item Uncached memory stores
  \end{itemize}
  But usually best left to compiler / HW.
  
\end{frame}


% =====================================================


\begin{frame}
  \frametitle{Matrix multiplication}

  \begin{itemize}
  \item This was a lot of stuff in a short time!
  \item Best way to digest it is try some things out
  \item First project: tune matrix-matrix multiply
  \item Due Sep 12 (about two weeks)
    \begin{itemize}
    \item Gives enough time to play with some ideas
    \item Not enough time for obsessive tuning to ruin lives
    \end{itemize}
  \item We encourage partners -- try to cross disciplines!
  \end{itemize}
\end{frame}


\begin{frame}
  \frametitle{Recommended strategy}

  \begin{itemize}
  \item Start with a small ``kernel'' multiply
    \begin{itemize}
    \item Maybe odd sizes, strange layouts -- just go fast!
    \item Intel compiler may do fine with simple-looking code
    \item Deserves its own timing rig
    \end{itemize}
  \item Use blocking to build up larger multiplies
  \item Will have to do something reasonable with edge blocks...
  \end{itemize}
  
\end{frame}

\begin{frame}
  \frametitle{References}

  \begin{itemize}
  \item My
    \href{http://www.cs.cornell.edu/~bindel/class/cs5220-f11/notes/serial-tuning.pdf}{serial
      tuning notes}.
  \item Ulrich Drepper,
    \href{http://people.redhat.com/drepper/cpumemory.pdf}{\em What
      Every Programmer Should Know About Memory}
  \item \href{http://www.intel.com/Assets/PDF/manual/248966.pdf}{Intel
    Optimization Manual}
  \item Hager and Wellein,
    \href{https://www.crcpress.com/Introduction-to-High-Performance-Computing-for-Scientists-and-Engineers/Hager-Wellein/p/book/9781439811924}{\em
      Intro
      to HPC for Scientists and Engineers}
  \item Goedecker and Hoisie, \href{http://epubs.siam.org/doi/book/10.1137/1.9780898718218}{\em Performance Optimization of
    Numerically Intensive Codes}
  \item Agner Fog's \href{http://www.agner.org/optimize/}{Software
    Optimization Manuals}
  \end{itemize}
\end{frame}


\end{document}
