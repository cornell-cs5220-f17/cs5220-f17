\documentclass{beamer}

% Should be documentclass beamer

\mode<presentation>
{
%  \usetheme[hideothersubsections]{PaloAlto}
  \usetheme{metropolis}
  \setbeamercovered{transparent}
}

\input{commonm}

\newcommand{\hdr}[2]{
  \title[CS 5220, Fall 2017]{CS 5220: #2}
  \author{David Bindel}
  \institute{Cornell}
  \date{#1}
}

\hdr{2017-08-24}{Performance basics}

\begin{document}


\begin{frame}
  \titlepage
\end{frame}


\begin{frame}
  \frametitle{Starting on the Soap Box}

  \begin{itemize}
  \item The goal is right enough, fast enough --- not flop/s.
  \item Performance is not all that matters.
    \begin{itemize}
    \item Portability, readability, debuggability matter too!
    \item Want to make intelligent trade-offs.
    \end{itemize}
  \item The road to good performance starts with a single core.
    \begin{itemize}
    \item Even single-core performance is hard.
    \item Helps to build on well-engineered libraries.
    \end{itemize}
  \item Parallel efficiency is hard!
    \begin{itemize}
    \item $p$ processors $\neq$ speedup of $p$
    \item Different algorithms parallelize differently.
    \item Speed vs a naive, untuned serial algorithm is cheating!
    \end{itemize}
  \end{itemize}
\end{frame}


\begin{frame}[fragile]
  \frametitle{The Cost of Computing}

  Consider a simple serial code:
  \begin{lstlisting}
    // Accumulate C += A*B for n-by-n matrices
    for (i = 0; i < n; ++i)
      for (j = 0; j < n; ++j)
        for (k = 0; k < n; ++k)
          C[i+j*n] += A[i+k*n] * B[k+j*n];
  \end{lstlisting}
  Simplest model:
  \begin{enumerate}
  \item Dominant cost is $2n^3$ flops (adds and multiplies)
  \item One flop per clock cycle
  \item Expected time is
    \[
    \mbox{Time (s)} \approx
    \frac{2n^3 \mbox{ flops}}
         {2.4 \cdot 10^9 \mbox{ cycle/s} \times 1 \mbox{ flop/cycle}}
    \]
  \end{enumerate}
  Problem: Model assumptions are wrong!
\end{frame}


\begin{frame}
  \frametitle{The Cost of Computing}

  Dominant cost is $2n^3$ flops (adds and multiplies)?
  \begin{itemize}
  \item Dominant cost is often memory traffic!
  \item Special case of a {\em communication cost}
  \item Two pieces to cost of fetching data
    \begin{description}
    \item[Latency] Time from operation start to first result (s)
    \item[Bandwidth] Rate at which data arrives (bytes/s)
    \end{description}
  \item Usually latency $\gg$ bandwidth$^{-1}$ $\gg$ time per flop
  \item Latency to L3 cache is 10s of ns, DRAM is $3$--$4 \times$
    slower
  \item Partial solution: caches (to discuss next time)
  \end{itemize}
  See: \href{https://gist.github.com/jboner/2841832}{Latency numbers every programmer should know}
\end{frame}


\begin{frame}
  \frametitle{The Cost of Computing}
  
  One flop per clock cycle? For cluster CPU cores:
  \[
  2 \frac{\mbox{flops}}{\mbox{FMA}} \times
  4 \frac{\mbox{FMA}}{\mbox{vector FMA}} \times
  2 \frac{\mbox{vector FMA}}{\mbox{cycle}} =
  16 \frac{\mbox{flops}}{\mbox{cycle}}
  \]
  Theoretical peak (one core) is
  \[
  \mbox{Time (s)} \approx
  \frac{2n^3 \mbox{ flops}}
       {2.4 \cdot 10^9 \mbox{ cycle/s} \times 16 \mbox{ flop/cycle}}
  \]

  \vspace{5mm}
  Makes DRAM latency look even worse!
  DRAM latency $\sim 100$ ns:
  \[
  100 \mbox{ ns} \times
  2.4 \frac{\mbox{cycle}}{\mbox{ns}} \times
  16 \frac{\mbox{flops}}{\mbox{cycle}} =
  3840 \mbox{ flops}
  \]

\end{frame}


\begin{frame}
  \frametitle{The Cost of Computing}

  Theoretical peak for matrix-matrix product (one core) is
  \[
    \mbox{Time (s)} \approx
    \frac{2n^3 \mbox{ flops}}
         {2.4 \cdot 10^9 \mbox{ cycle/s} \times 16 \mbox{ flop/cycle}}
  \]
  For 12 core node, theoretical peak is $12 \times$ faster.       
  \begin{itemize}
  \item But lose orders of magnitude if too many memory refs
  \item And getting full vectorization is also not easy!
  \item We'll talk more about (single-core) arch next week
  \end{itemize}

\end{frame}


\begin{frame}
  \frametitle{The Cost of Computing}

  Sanity check: What is the theoretical peak of a Xeon Phi 5110P
  accelerator? \\
  \hfill \href{https://en.wikipedia.org/wiki/Xeon_Phi}{Wikipedia to the rescue!}
\end{frame}


\begin{frame}
  \frametitle{The Cost of Computing}

  What to take away from this performance modeling example?
  \begin{itemize}
  \item Start with a simple model
    \begin{itemize}
    \item Simplest model is asymptotic complexity (e.g.~$O(n^3)$ flops)
    \item Counting {\em every} detail just complicates life
    \item But we want enough detail to predict something
    \end{itemize}
  \item Watch out for hidden costs
    \begin{itemize}
    \item Flops are not the only cost!
    \item Memory/communication costs are often killers
    \item Integer computation may play a role as well
    \end{itemize}
  \item Account for instruction-level parallelism, too!
  \end{itemize}
  And we haven't even talked about more than one core yet!
\end{frame}


\begin{frame}
  \frametitle{The Cost of (Parallel) Computing}

  Simple model:
  \begin{itemize}
  \item Serial task takes time $T$ (or $T(n)$)
  \item Deploy $p$ processors
  \item Parallel time is $T(n)/p$
  \end{itemize}
  ... and you should be suspicious by now!
  
\end{frame}


\begin{frame}
  \frametitle{The Cost of (Parallel) Computing}

  Why is parallel time not $T/p$?
  \begin{itemize}
  \item {\bf Overheads:} Communication, synchronization, extra
    computation and memory overheads
  \item {\bf Intrinsically serial} work
  \item {\bf Idle time} due to synchronization
  \item {\bf Contention} for resources
  \end{itemize}
  We will talk about all of these in more detail.
  
\end{frame}


\begin{frame}
  \frametitle{Quantifying Parallel Performance}
  
  \begin{itemize}
  \item Starting point: good {\em serial} performance
  \item Scaling study: compare parallel to serial time as a function of number of processors ($p$)
    \begin{align*}
      \mbox{Speedup} &= \frac{\mbox{Serial time}}{\mbox{Parallel time}} \\[2mm]
      \mbox{Efficiency} &= \frac{\mbox{Speedup}}{p}
    \end{align*}
  \item
    Ideally, speedup = $p$.  
    Usually, speedup $ < p$.
  \item Barriers to perfect speedup
    \begin{itemize}
    \item Serial work (Amdahl's law)
    \item Parallel overheads (communication, synchronization)
    \end{itemize}
  \end{itemize}
\end{frame}


\begin{frame}
  \frametitle{Amdahl's Law}

  Parallel scaling study where some serial code remains:
  \begin{align*}
    p = & \mbox{ number of processors} \\
    s = & \mbox{ fraction of work that is serial} \\
    t_s = & \mbox{ serial time} \\
    t_p = & \mbox{ parallel time} \geq s t_s + (1-s) t_s / p
  \end{align*}

  \vspace{2mm}
  Amdahl's law:
  \[
    \mbox{Speedup} = 
      \frac{t_s}{t_p} = \frac{1}{s + (1-s) / p} > \frac{1}{s}
  \]

  \vspace{5mm}
  So $1\%$ serial work $\implies$ max speedup < $100 \times$,
  regardless of $p$.
\end{frame}


\begin{frame}
  \frametitle{Strong and weak scaling}

  Ahmdahl looks bad!  But two types of scaling studies:
  \begin{description}
  \item[Strong scaling] Fix problem size, vary $p$
  \item[Weak scaling] Fix work per processor, vary $p$
  \end{description}

  \vspace{5mm}
  For weak scaling, study {\em scaled speedup}
  \[
    S(p) =
    \frac{T_{\mbox{serial}}(n(p))}{T_{\mbox{parallel}}(n(p), p)}
  \]
  Gustafson's Law:
  \[
    S(p) \leq p - \alpha(p-1)
  \]
  where $\alpha$ is the fraction of work that is serial.
\end{frame}


\begin{frame}
  \frametitle{Pleasing Parallelism}

  A task is ``pleasingly parallel'' (aka ``embarrassingly parallel'')
  if it requires very little coordination, for example:
  \begin{itemize}
  \item Monte Carlo computations with many independent trials
  \item Big data computations mapping many data items independently
  \end{itemize}
  Result is ``high-throughput'' computing -- easy to get impressive
  speedups!  Says nothing about hard-to-parallelize tasks.

\end{frame}


\begin{frame}[fragile]
  \frametitle{Dependencies}

  Main pain point: {\em dependency} between computations
  \begin{lstlisting}
    a = f(x)
    b = g(x)
    c = h(a,b)
  \end{lstlisting}
  Compute {\tt a} and {\tt b} in parallel, but finish both
  before {\tt c}! \\
  Limits amount of parallel work available.

  This is a true dependency (read-after-write).  Also have false
  dependencies (write-after-read and write-after-write) that can be
  dealt with more easily.
\end{frame}


\begin{frame}
  \frametitle{Granularity}

  \begin{itemize}
  \item Coordination is expensive --- including parallel start/stop!
  \item Need to do enough work to amortize parallel costs
  \item Not enough to have parallel work, need big chunks!
  \item How big the chunks must be depends on the machine.
  \end{itemize}
\end{frame}


\begin{frame}
  \frametitle{Patterns and Benchmarks}

  If your task is not pleasingly parallel, you ask:
  \begin{itemize}
  \item What is the best performance I reasonably expect?
  \item How do I get that performance?
  \end{itemize}
  Look at examples somewhat like yours --
  a {\em parallel pattern} -- and maybe seek an informative
  benchmark.  Better yet: reduce to a previously
  well-solved problem (build on tuned {\em kernels}).

  \vspace{1cm}
  NB: Easy to pick uninformative benchmarks and go astray.
\end{frame}

\end{document}
