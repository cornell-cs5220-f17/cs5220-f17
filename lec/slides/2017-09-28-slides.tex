\documentclass{beamer}

% Should be documentclass beamer

\mode<presentation>
{
%  \usetheme[hideothersubsections]{PaloAlto}
  \usetheme{metropolis}
  \setbeamercovered{transparent}
}

\input{commonm}

\newcommand{\hdr}[2]{
  \title[CS 5220, Fall 2017]{CS 5220: #2}
  \author{David Bindel}
  \institute{Cornell}
  \date{#1}
}


\hdr{2017-09-28}{Shared memory programming}

\begin{document}

\begin{frame}
  \titlepage
\end{frame}


\begin{frame}
  \frametitle{OpenMP: Open spec for MultiProcessing}

  \begin{itemize}
  \item Standard API for multi-threaded code
    \begin{itemize}
    \item Only a spec --- multiple implementations
    \item Lightweight syntax
    \item C or Fortran (with appropriate compiler support)
    \end{itemize}
  \item High level:
    \begin{itemize}
    \item Preprocessor/compiler directives (80\%)
    \item Library calls (19\%)
    \item Environment variables (1\%)
    \end{itemize}
  \item Basic syntax: {\tt \#pragma omp {\it construct} [{\it clause} ...]}
    \begin{itemize}
    \item Usually affects structured block (one way in/out)
    \item OK to have {\tt exit()} in such a block
    \end{itemize}
  \end{itemize}
\end{frame}


\begin{frame}
  \frametitle{Last time}

  \begin{itemize}
  \item Environmental inquiries with {\tt omp\_get\_*} functions
  \item Creating parallel regions with {\tt \#pragma omp parallel}
  \item Annotations for variables (shared, private, reduction)
  \item Synchronization via critical sections, atomic ops, barriers
  \item Today: Work sharing, tasks, and some examples
  \end{itemize}
\end{frame}


\begin{frame}
  \frametitle{Work sharing}

  {\em Work sharing} constructs split work across a team
  \begin{itemize}
  \item Parallel {\tt for}: split by loop iterations
  \item {\tt sections}: non-iterative tasks
  \item {\tt single}: only one thread executes (synchronized)
  \item {\tt master}: master executes, others skip (no sync)
  \end{itemize}
\end{frame}


\begin{frame}[fragile]
  \frametitle{Parallel iteration}

  Idea: Map {\bf independent} iterations onto different threads
\begin{lstlisting}
  #pragma omp parallel for
  for (int i = 0; i < N; ++i)
      a[i] += b[i];

  #pragma omp parallel
  {
      // Stuff can go here...
      #pragma omp for    
      for (int i = 0; i < N; ++i)
          a[i] += b[i];
  }
\end{lstlisting}
  Implicit barrier at end of loop (unless {\tt nowait} clause)
\end{frame}


\begin{frame}[fragile]
  \frametitle{Parallel iteration}

  The iteration can also go across a higher-dim index set
\begin{lstlisting}
  #pragma omp parallel for collapse(2)    
  for (int i = 0; i < N; ++i)
    for (int j = 0; j < M; ++j)
      a[i*M+j] = foo(i,j);
\end{lstlisting}
\end{frame}


\begin{frame}
  \frametitle{Restrictions}

  \begin{itemize}
  \item {\tt for} loop must be in ``canonical form''
    \begin{itemize}
    \item Loop var is an integer, pointer, random access iterator (C++)
    \item Test compares loop var to loop-invariant expression
    \item Increment or decrement by a loop-invariant expression
    \item No code between loop starts in {\tt collapse} set
    \item Needed to split iteration space (maybe in advance)
    \end{itemize}
  \item Iterations should be independent
    \begin{itemize}
    \item Compiler may not stop you if you screw this up!
    \end{itemize}
  \item Iterations may be assigned out-of-order on one thread!
    \begin{itemize}
    \item Unless the loop is declared {\tt monotonic}
    \end{itemize}
  \end{itemize}
\end{frame}


\begin{frame}[fragile]
  \frametitle{Reduction loops}

  How might we parallelize something like this?
  \begin{lstlisting}
    double sum = 0;
    for (int i = 0; i < N; ++i)
        sum += big_hairy_computation(i);
  \end{lstlisting}
\end{frame}


\begin{frame}[fragile]
  \frametitle{Reduction loops}

  How might we parallelize something like this?
  \begin{lstlisting}
    double sum = 0;
    #pragma omp parallel for reduction(+:sum)
    for (int i = 0; i < N; ++i)
        sum = big_hairy_computation(i);
  \end{lstlisting}
\end{frame}


\begin{frame}[fragile]
  \frametitle{Ordered}

  OK, what about something like this?
  \begin{lstlisting}
    for (int i = 0; i < N; ++i) {
        int result = big_hairy_computation(i);
        add_to_queue(q, result);
    }
  \end{lstlisting}
  Work is {\em mostly} independent, but not wholly.
\end{frame}


\begin{frame}[fragile]
  \frametitle{Ordered}

  Solution: {\tt ordered} directive in loop with {\tt ordered clause}
  \begin{lstlisting}
    #pragma omp parallel for ordered
    for (int i = 0; i < N; ++i) {
        int result = big_hairy_computation(i);
        #pragma ordered
        add_to_queue(q, result);
    }
  \end{lstlisting}
  Ensures the {\tt ordered} code executes in loop order.
\end{frame}


\begin{frame}[fragile]
  \frametitle{SIMD loops}

  As of OpenMP 4.0:
\begin{lstlisting}
    #pragma omp parallel simd reduction(+:sum) aligned(a:64)
    for (int i = 0; i < N; ++i) {
        a[i] = b[i] * c[i];
        sum = sum + a[i];
    }
\end{lstlisting}
  Can also declare vectorized functions: \\
\begin{lstlisting}
    #pragma omp declare simd
    float myfunc(float a, float b, float c)
    {
        return a*b + c;
    }
\end{lstlisting}
\end{frame}


\begin{frame}
  \frametitle{Other parallel work divisions}
  
  \begin{itemize}
  \item {\tt sections}: like cobegin/coend
  \item {\tt single}: do only in one thread (e.g. I/O)
  \item {\tt master}: do only in master thread; others skip
  \end{itemize}
\end{frame}


\begin{frame}[fragile]
  \frametitle{Sections}

\begin{lstlisting}
    #pragma omp parallel
    {
        #pragma omp sections nowait
        {
            #pragma omp section
            do_something();

            #pragma omp section
            and_something_else();

            #pragma omp section
            and_this_too();
            // No implicit barrier here
        }
        // Implicit barrier here
    }
\end{lstlisting}
{\tt sections nowait} to kill barrier.
\end{frame}


\begin{frame}
  \frametitle{Task-based parallelism}

  \begin{itemize}
  \item Work-sharing so far is rather limited
    \begin{itemize}
    \item Work cannot be produced/consumed dynamically
    \item Fine for data parallel array processing...
    \item ... but what about tree walks and such?
    \end{itemize}
  \item Alternate approach (OpenMP 3.0+): Tasks
  \end{itemize}
\end{frame}


\begin{frame}
  \frametitle{Tasks}

  Task involves:
  \begin{itemize}
  \item Task construct: {\tt task} directive plus structured block
  \item Task: Task construct + data
  \end{itemize}
  Tasks are handled by run time, complete at barriers or {\tt taskwait}.
\end{frame}


\begin{frame}[fragile]
  \frametitle{Example: List traversal}

\begin{lstlisting}
#pragma omp parallel
{
    #pragma omp single nowait
    {
        for (link_t* link = head; link; link = link->next)
            #pragma omp task firstprivate(link)
            process(link);
    }
    // Implicit barrier
}
\end{lstlisting}
One thread generates tasks, others execute them.
\end{frame}


\begin{frame}[fragile]
  \frametitle{Example: Tree traversal}

\begin{lstlisting}
int tree_max(node_t* n)
{
    int lmax, rmax;
    if (n->is_leaf)
        return n->value;
    
    #pragma omp task shared(lmax)
        lmax = tree_max(n->l);
    #pragma omp task shared(rmax)
        rmax = tree_max(n->l);
    #pragma omp taskwait

    return max(lmax, rmax);
}
\end{lstlisting}
The {\tt taskwait} waits for all child tasks.
\end{frame}


\begin{frame}[fragile]
  \frametitle{Task dependencies}

  What happens if one task produces what another needs?
\begin{lstlisting}
    #pragma omp task depend(out:x)
    x = foo();
    #pragma omp task depend(in:x)
    y = bar(x);
\end{lstlisting}

\end{frame}


\begin{frame}
  \frametitle{Some examples (at board)}

  What are different ways to organize these:
  \begin{itemize}
  \item Dot product?
  \item Monte Carlo computation with adaptive termination?
  \item Wave equation time stepper?
  \end{itemize}
\end{frame}

\end{document}




\begin{frame}[fragile]
  \frametitle{Toy problem: actual code (OpenMP)}
\end{frame}


\begin{frame}[fragile]
  \frametitle{Toy problem: actual code (OpenMP)}

  A practical aside...
  \begin{itemize}
  \item GCC 4.3+ has OpenMP support by default
    \begin{itemize}
    \item Earlier versions may support (e.g. latest Xcode {\tt gcc-4.2})
    \item GCC 4.4 (prerelease) for my laptop has buggy support!
    \item {\tt -O3 -fopenmp} == death of an afternoon
    \end{itemize}
  \item Need {\tt -fopenmp} for both compile and link lines
\begin{lstlisting}
gcc -c -fopenmp foo.c
gcc -o -fopenmp mycode.x foo.o
\end{lstlisting}
  \end{itemize}
\end{frame}


\begin{frame}
  \frametitle{Parallel loops}

\begin{center}
\includegraphics[width=\textwidth]{omploop.pdf}
\end{center}

\begin{itemize}
\item
  Independent loop body? At least order doesn't matter%
\footnote{If order matters, there's an {\tt ordered} modifier.}.
\item
  Partition index space among threads
\item 
  Implicit barrier at end (except with {\tt nowait})
\end{itemize}

\end{frame}


\begin{frame}[fragile]
  \frametitle{Parallel loops}

\begin{lstlisting}
/* Compute dot of x and y of length n */
int i, tid;
double my_dot, dot = 0;
#pragma omp parallel \
        shared(dot,x,y,n) \
        private(i,my_dot)
{
  tid = omp_get_thread_num();
  my_dot = 0;
 
  #pragma omp for
  for (i = 0; i < n; ++i)
    my_dot += x[i]*y[i];

  #pragma omp critical
  dot += my_dot;
}
\end{lstlisting}

\end{frame}


\begin{frame}[fragile]
  \frametitle{Parallel loops}

\begin{lstlisting}
/* Compute dot of x and y of length n */
int i, tid;
double dot = 0;
#pragma omp parallel \
        shared(x,y,n) \
        private(i) \
        reduction(+:dot)
{
  #pragma omp for
  for (i = 0; i < n; ++i)
    dot += x[i]*y[i];
}
\end{lstlisting}


\begin{frame}
  \frametitle{Parallel loop scheduling}
  
  Partition index space different ways:
  \begin{itemize}
  \item {\tt static[(chunk)]}: decide at start of loop;
    default chunk is {\tt n/nthreads}.  Lowest overhead,
    most potential load imbalance.
  \item {\tt dynamic[(chunk)]}: each thread takes {\tt chunk}
    iterations when it has time; default {\tt chunk} is 1.
    Higher overhead, but automatically balances load.
  \item {\tt guided}: take chunks of size unassigned iterations/threads;
    chunks get smaller toward end of loop.  Somewhere between
    {\tt static} and {\tt dynamic}.
  \item {\tt auto}: up to the system!
  \end{itemize}
  Default behavior is implementation-dependent.

\end{frame}


\end{document}
