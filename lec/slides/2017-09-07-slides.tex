\documentclass{beamer}

% Should be documentclass beamer

\mode<presentation>
{
%  \usetheme[hideothersubsections]{PaloAlto}
  \usetheme{metropolis}
  \setbeamercovered{transparent}
}

\usepackage{amsfonts}
\usepackage{amsmath}
\usepackage{amssymb}
\usepackage{color}
\usepackage{tikz}
\usepackage{pgfplots}
\usepackage{listings}
\usepackage{courier}
%\usepackage[utf8]{inputenc}
%\usepackage[russian]{babel}

\lstset{
  numbers=left,
  basicstyle=\ttfamily\footnotesize,
  numberstyle=\tiny\color{gray},
  stepnumber=1,
  numbersep=10pt,
}

\newcommand{\iu}{\ensuremath{\mathrm{i}}}
\newcommand{\bbR}{\mathbb{R}}
\newcommand{\bbC}{\mathbb{C}}
\newcommand{\calV}{\mathcal{V}}
\newcommand{\calW}{\mathcal{W}}
\newcommand{\macheps}{\epsilon_{\mathrm{mach}}}
\newcommand{\matlab}{\textsc{Matlab}}

\newcommand{\ddiag}{\operatorname{diag}}
\newcommand{\fl}{\operatorname{fl}}
\newcommand{\nnz}{\operatorname{nnz}}
\newcommand{\tr}{\operatorname{tr}}
\renewcommand{\vec}{\operatorname{vec}}

\newcommand{\vertiii}[1]{{\left\vert\kern-0.25ex\left\vert\kern-0.25ex\left\vert #1
    \right\vert\kern-0.25ex\right\vert\kern-0.25ex\right\vert}}
\newcommand{\ip}[2]{\langle #1, #2 \rangle}
\newcommand{\ipx}[2]{\left\langle #1, #2 \right\rangle}
\newcommand{\order}[1]{O( #1 )}

\newcommand{\kron}{\otimes}


\newcommand{\hdr}[2]{
  \title[CS 5220, Fall 2017]{CS 5220: #2}
  \author{David Bindel}
  \date{#1}
}


\hdr{2017-09-07}{Parallel machines and models}


\begin{document}

\begin{frame}
  \titlepage
\end{frame}


\begin{frame}
  \frametitle{Why clusters?}

  \begin{itemize}
  \item Clusters of SMPs are everywhere
    \begin{itemize}
    \item Commodity hardware -- economics!  Even supercomputers
      now use commodity CPUs (+ specialized interconnects).
    \item Relatively simple to set up and administer (?)
    \end{itemize}
  \item But still costs room, power, ...
  \item Economy of scale $\implies$ clouds?
    \begin{itemize}
    \item Amazon and MS now have HPC instances (GCP, too)
    \item Microsoft has Infiniband connected instances
    \item Several bare-metal HPC/cloud providers
    \item Lots of interesting challenges here
    \end{itemize}
  \end{itemize}
\end{frame}


\begin{frame}
  \frametitle{Cluster structure}

  Consider:
  \begin{itemize}
  \item Each core has vector parallelism
  \item Each chip has six cores, shares memory with others
  \item Each box has two chips, shares memory
  \item Each box has two Xeon Phi accelerators
  \item Eight instructional nodes, communicate via Ethernet
  \end{itemize}
  How did we get here? Why this type of structure?  And how does the
  programming model match the hardware?

\end{frame}


\begin{frame}
  \frametitle{Parallel computer hardware}
  
  Physical machine has {\em processors}, {\em memory}, {\em interconnect}.
  \begin{itemize}
  \item Where is memory physically?
  \item Is it attached to processors?
  \item What is the network connectivity?
  \end{itemize}
\end{frame}


\begin{frame}
  \frametitle{Parallel programming model}
  
  Programming {\em model} through languages, libraries.
  \begin{itemize}
  \item
    Control
    \begin{itemize}
      \item How is parallelism created?
      \item What ordering is there between operations?
    \end{itemize}
  \item
    Data
    \begin{itemize}
      \item What data is private or shared?
      \item How is data logically shared or communicated?
    \end{itemize}
  \item
    Synchronization
    \begin{itemize}
      \item What operations are used to coordinate?
      \item What operations are atomic?
    \end{itemize}
  \item
    Cost: how do we reason about each of above?
  \end{itemize}
\end{frame}


\begin{frame}
  \frametitle{Simple example}

  Consider dot product of $x$ and $y$.
  \begin{itemize}
    \item Where do arrays $x$ and $y$ live?  One CPU?  Partitioned?
    \item Who does what work?
    \item How do we combine to get a single final result?
  \end{itemize}
\end{frame}


\begin{frame}
  \frametitle{Shared memory programming model}
  
  Program consists of {\em threads} of control.
  \begin{itemize}
  \item Can be created dynamically
  \item Each has private variables (e.g.~local)
  \item Each has shared variables (e.g.~heap)
  \item Communication through shared variables
  \item Coordinate by synchronizing on variables
  \item Examples: OpenMP, pthreads
  \end{itemize}
\end{frame}


\begin{frame}
  \frametitle{Shared memory dot product}
  
  Dot product of two $n$ vectors on $p \ll n$ processors:
  \begin{enumerate}
  \item Each CPU evaluates partial sum ($n/p$ elements, local)
  \item Everyone tallies partial sums
  \end{enumerate}
  Can we go home now?
\end{frame}


\begin{frame}
  \frametitle{Race condition}

  A {\em race condition}:
  \begin{itemize}
  \item Two threads access same variable, at least one write.
  \item Access are concurrent -- no ordering guarantees
    \begin{itemize}
    \item Could happen simultaneously!
    \end{itemize}
  \end{itemize}
  Need synchronization via lock or barrier.
\end{frame}


\begin{frame}[fragile]
  \frametitle{Race to the dot}

  Consider {\tt S += partial\_sum} on 2 CPU:
  \begin{itemize}
  \item P1: Load {\tt S}
  \item P1: Add {\tt partial\_sum}
  \item P2: Load {\tt S}
  \item P1: Store new {\tt S}
  \item P2: Add {\tt partial\_sum}
  \item P2: Store new {\tt S}
  \end{itemize}
\end{frame}

\begin{frame}
  \frametitle{Shared memory dot with locks}

  Solution: consider {\tt S += partial\_sum} a {\em critical section}
  \begin{itemize}
  \item Only one CPU at a time allowed in critical section
  \item Can violate invariants locally
  \item Enforce via a lock or mutex (mutual exclusion variable)
  \end{itemize}
  
  \vspace{5mm}
  Dot product with mutex:
  \begin{enumerate}
  \item Create global mutex {\tt l}
  \item Compute {\tt partial\_sum}
  \item Lock {\tt l}
  \item {S += partial\_sum}
  \item Unlock {\tt l}
  \end{enumerate}

\end{frame}


\begin{frame}
  \frametitle{Shared memory with barriers}
  
  \begin{itemize}
  \item Lots of sci codes have phases (e.g. time steps)
  \item Communication only needed at end of phases
  \item Idea: synchronize on end of phase with {\em barrier}
    \begin{itemize}
    \item More restrictive (less efficient?) than small locks
    \item But easier to think through!  (e.g. less chance of deadlocks)
    \end{itemize}
  \item Sometimes called {\em bulk synchronous programming}
  \end{itemize}

\end{frame}


\begin{frame}
  \frametitle{Shared memory machine model}

  \begin{itemize}
  \item Processors and memories talk through a bus
  \item Symmetric Multiprocessor (SMP)
  \item Hard to scale to lots of processors (think $\leq 32$)
    \begin{itemize}
      \item Bus becomes bottleneck
      \item {\em Cache coherence} is a pain
    \end{itemize}
  \item Example: Six-core chips on cluster
  \end{itemize}
\end{frame}


\begin{frame}
  \frametitle{Multithreaded processor machine}
  
  \begin{itemize}
  \item Maybe threads > processors!
  \item Idea: Switch threads on long latency ops.
  \item Called {\em hyperthreading} by Intel
  \item Cray MTA was an extreme example
  \end{itemize}
\end{frame}


\begin{frame}
  \frametitle{Distributed shared memory}
 
  \begin{itemize}
  \item Non-Uniform Memory Access (NUMA)
  \item Can {\em logically} share memory while {\em physically} distributing
  \item Any processor can access any address
  \item Cache coherence is still a pain
  \item Example: SGI Origin (or multiprocessor nodes on cluster)
  \item Many-core accelerators tend to be NUMA as well
  \end{itemize}
\end{frame}


\begin{frame}
  \frametitle{Message-passing programming model}

  \begin{itemize}
  \item Collection of named processes
  \item Data is {\em partitioned}
  \item Communication by send/receive of explicit message
  \item Lingua franca: MPI (Message Passing Interface)
  \end{itemize}
\end{frame}


\begin{frame}
  \frametitle{Message passing dot product: v1}

  \begin{minipage}{0.45\textwidth}
    Processor 1:
    \begin{enumerate}
    \item Partial sum s1
    \item Send s1 to P2
    \item Receive s2 from P2
    \item s = s1 + s2
    \end{enumerate}
  \end{minipage}
  \begin{minipage}{0.45\textwidth}
    Processor 2:
    \begin{enumerate}
    \item Partial sum s2
    \item Send s2 to P1
    \item Receive s1 from P1
    \item s = s1 + s2
    \end{enumerate}
  \end{minipage}

  \vspace{1cm}
  What could go wrong?  Think of phones vs letters...

\end{frame}


\begin{frame}
  \frametitle{Message passing dot product: v1}

  \begin{minipage}{0.45\textwidth}
    Processor 1:
    \begin{enumerate}
    \item Partial sum s1
    \item Send s1 to P2
    \item Receive s2 from P2
    \item s = s1 + s2
    \end{enumerate}
  \end{minipage}
  \begin{minipage}{0.45\textwidth}
    Processor 2:
    \begin{enumerate}
    \item Partial sum s2
    \item Receive s1 from P1
    \item Send s2 to P1
    \item s = s1 + s2
    \end{enumerate}
  \end{minipage}

  \vspace{1cm}
  Better, but what if more than two processors?

\end{frame}


\begin{frame}
  \frametitle{MPI: the de facto standard}

  \begin{itemize}
  \item Pro: {\em Portability}
  \item Con: least-common-denominator for mid 80s
  \end{itemize}
  The ``assembly language'' (or C?) of parallelism... \\
  \hspace{5mm} but, alas, assembly language can be high performance.

\end{frame}


\begin{frame}
  \frametitle{Distributed memory machines}
  
  \begin{itemize}
  \item Each node has local memory
    \begin{itemize}
    \item ... and no direct access to memory on other nodes
    \end{itemize}
  \item Nodes communicate via network interface
  \item Example: our cluster!
  \item Other examples: IBM SP, Cray T3E
  \end{itemize}
\end{frame}


\begin{frame}
  \frametitle{The story so far}

  \begin{itemize}
  \item Even {\em serial} performance is a complicated function of
        the underlying architecture and memory system.  We need to
        understand these effects in order to design data structures
        and algorithms that are fast on modern machines.  Good
        serial performance is the basis for good parallel performance.
  \item {\em Parallel} performance is additionally complicated by
        communication and synchronization overheads, and by how
        much parallel work is available.  If a small fraction of the
        work is completely serial, Amdahl's law bounds the speedup,
        independent of the number of processors.
  \item We have discussed serial architecture and some of the basics
        of parallel machine models and programming models.
  \item Now we want to describe how to think about the shape of
        parallel algorithms for some scientific applications.
  \end{itemize}
\end{frame}


\begin{frame}
  \frametitle{Reminder: what do we want?}

  \begin{itemize}
  \item High-level: solve big problems fast
  \item Start with good {\em serial} performance
  \item Given $p$ processors, could then ask for
    \begin{itemize}
    \item Good {\em speedup}: $p^{-1}$ times serial time
    \item Good {\em scaled speedup}: $p$ times the work in same time
    \end{itemize}
  \item Easiest to get good speedup from cruddy serial code!
  \end{itemize}
\end{frame}


\begin{frame}
  \frametitle{Parallelism and locality}

  \begin{itemize}
  \item Real world exhibits {\em parallelism} and {\em locality}
    \begin{itemize}
    \item Particles, people, etc function independently
    \item Nearby objects interact more strongly than distant ones
    \item Can often simplify dependence on distant objects
    \end{itemize}
  \item Can get more parallelism / locality through model
    \begin{itemize}
    \item Limited range of dependency between adjacent time steps
    \item Can neglect or approximate far-field effects
    \end{itemize}
  \item Often get parallism at multiple levels
    \begin{itemize}
    \item Hierarchical circuit simulation
    \item Interacting models for climate
    \item Parallelizing individual experiments in MC or optimization
    \end{itemize}
  \end{itemize}
\end{frame}


\begin{frame}
  \frametitle{Basic styles of simulation}

  \begin{itemize}
  \item Discrete event systems (continuous or discrete time)
    \begin{itemize}
    \item Game of life, logic-level circuit simulation
    \item Network simulation
    \end{itemize}
  \item Particle systems
    \begin{itemize}
    \item Billiards, electrons, galaxies, ...
    \item Ants, cars, ...?
    \end{itemize}
  \item Lumped parameter models (ODEs)
    \begin{itemize}
    \item Circuits (SPICE), structures, chemical kinetics
    \end{itemize}
  \item Distributed parameter models (PDEs / integral equations)
    \begin{itemize}
    \item Heat, elasticity, electrostatics, ...
    \end{itemize}
  \end{itemize}
  Often more than one type of simulation appropriate. \\
  Sometimes more than one at a time!

\end{frame}

\end{document}
