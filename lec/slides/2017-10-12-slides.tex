\documentclass{beamer}

% Should be documentclass beamer

\mode<presentation>
{
%  \usetheme[hideothersubsections]{PaloAlto}
  \usetheme{metropolis}
  \setbeamercovered{transparent}
}

\usepackage{amsfonts}
\usepackage{amsmath}
\usepackage{amssymb}
\usepackage{color}
\usepackage{tikz}
\usepackage{pgfplots}
\usepackage{listings}
\usepackage{courier}
%\usepackage[utf8]{inputenc}
%\usepackage[russian]{babel}

\lstset{
  numbers=left,
  basicstyle=\ttfamily\footnotesize,
  numberstyle=\tiny\color{gray},
  stepnumber=1,
  numbersep=10pt,
}

\newcommand{\iu}{\ensuremath{\mathrm{i}}}
\newcommand{\bbR}{\mathbb{R}}
\newcommand{\bbC}{\mathbb{C}}
\newcommand{\calV}{\mathcal{V}}
\newcommand{\calW}{\mathcal{W}}
\newcommand{\macheps}{\epsilon_{\mathrm{mach}}}
\newcommand{\matlab}{\textsc{Matlab}}

\newcommand{\ddiag}{\operatorname{diag}}
\newcommand{\fl}{\operatorname{fl}}
\newcommand{\nnz}{\operatorname{nnz}}
\newcommand{\tr}{\operatorname{tr}}
\renewcommand{\vec}{\operatorname{vec}}

\newcommand{\vertiii}[1]{{\left\vert\kern-0.25ex\left\vert\kern-0.25ex\left\vert #1
    \right\vert\kern-0.25ex\right\vert\kern-0.25ex\right\vert}}
\newcommand{\ip}[2]{\langle #1, #2 \rangle}
\newcommand{\ipx}[2]{\left\langle #1, #2 \right\rangle}
\newcommand{\order}[1]{O( #1 )}

\newcommand{\kron}{\otimes}


\newcommand{\hdr}[2]{
  \title[CS 5220, Fall 2017]{CS 5220: #2}
  \author{David Bindel}
  \date{#1}
}


\hdr{2017-10-12}{VMs, containers, and clouds}

\begin{document}

\begin{frame}
  \titlepage
\end{frame}

% Book pointer
% https://cornell-ssw.github.io/meetings/2016-10-31#google-cloud-platform

% Overview: What is a cloud?  Providers, use cases
% Virtualized vs bare metal
% Containerization
% 


\begin{frame}
  \frametitle{Cloud vs~HPC}

  Is the cloud becoming a supercomputer?
  \begin{itemize}
  \item What does this even mean?
  \item Cloud $\approx$ resources for rent
    \begin{itemize}
    \item Compute cycles and raw bits, or something higher level?
    \item Bare metal or virtual machines?
    \item On demand, behind a queue?
    \end{itemize}
  \item Typically engineered for {\em different loads}
    \begin{itemize}
    \item Cloud: high utilization, services
    \item Super: a few users, big programs
    \end{itemize}
  \item But the picture is complicated...
  \end{itemize}
\end{frame}


\begin{frame}
  \frametitle{Choosing a platform}
\end{frame}


\begin{frame}
  \frametitle{Questions to ask}

  \begin{itemize}
  \item What type of workload do I have?
    \begin{itemize}
    \item Big memory but modest core count?
    \item Embarassingly parallel?
    \item GPU friendly?
    \end{itemize}
  \item How much data?  Data transfer is not always free!
  \item How will I interact with the system?  SSH alone?  GUIs?  Web?
  \item What about licensed software?
  \end{itemize}
\end{frame}


\begin{frame}
  \frametitle{Standard options beyond the laptop}

  \begin{itemize}
  \item Local clusters and servers
  \item Public cloud VMs (Amazon, Google, Azure)
    \begin{itemize}
    \item Can pay money or write proposal for credits
    \end{itemize}
  \item Public cloud bare metal (Nimbix, Sabalcore, PoD)
    \begin{itemize}
    \item Good if bare-metal parallel performance an issue
    \item Might want to compare to
      \href{https://www.cac.cornell.edu/}{CAC offerings}
    \end{itemize}
  \item Supercomputer (\href{https://www.xsede.org/}{XSEDE},
    \href{http://science.energy.gov/ascr/facilities/accessing-ascr-facilities/}{DOE})
  \end{itemize}
\end{frame}


\begin{frame}
  \frametitle{Topics du jour}

  \begin{itemize}
  \item Virtualization: supporting high utilization
  \item Containers: isolation without performance hits
  \item XaaS: the prevailing buzzword soup
  \end{itemize}
\end{frame}


\begin{frame}
  \frametitle{Virtualization}

  \begin{quote}
    All problems in computer science can be solved by another level of
    indirection. \\
    \hfill -- David Wheeler
  \end{quote}
\end{frame}


\begin{frame}
  \frametitle{From physical to logical}

  \begin{itemize}
  \item OS: Share HW resources between processes
    \begin{itemize}
    \item Provides processes with HW abstraction
    \end{itemize}
  \item Hypervisor: Share HW resources between virtual machiens
    \begin{itemize}
    \item Each VM has independent OS, utilities, libraries
    \item Sharing HW across VMs improves utilization
    \item Separating VM from HW improves portability
    \end{itemize}
  \end{itemize}
  Sharing HW across VMs is key to Amazon, Azure, Google clouds.
\end{frame}


\begin{frame}
  \frametitle{The Virtual Machine: CPU + memory}

  \begin{itemize}
  \item Sharing across processes with same OS is old
    \begin{itemize}
    \item OS-supported pre-emptive multi-tasking
    \item Virtual memory abstractions with HW support
      \begin{itemize}
      \item Page tables, TLB
      \end{itemize}
    \end{itemize}
  \item Sharing HW between systems is newer
    \begin{itemize}
    \item Today: CPU virtualization with near zero overhead
      \begin{itemize}
      \item Really?  Cache effects may be an issue
      \end{itemize}
    \item Backed by extended virtual memory support
      \begin{itemize}
      \item DMA remapping, extended page tables
      \end{itemize}
    \end{itemize}
  \end{itemize}
\end{frame}


\begin{frame}
  \frametitle{The Virtual Machine: Storage}

  \begin{itemize}
  \item Network attached storage around for a long time
  \item Modern clouds provide a blizzard of storage options
  \item SSD-enabled machines increasingly common
  \end{itemize}
\end{frame}


\begin{frame}
  \frametitle{The Virtual Machine: Network}

  \begin{itemize}
  \item Hard to get full-speed access via VM!
    \begin{itemize}
    \item Issue: Sharing peripherals with direct memory access?
    \item Issue: Force to go through TCP, or go lower?
    \end{itemize}
  \item HW support is improving (e.g.~SR-IOV standards)
  \item Still a potential pain point
  \end{itemize}
\end{frame}


\begin{frame}
  \frametitle{The Virtual Machine: Accelerators?}

  I don't understand how these would be virtualized! \\
  \href{https://aws.amazon.com/ec2/elastic-gpus/}{But I know people are doing it.}
\end{frame}


\begin{frame}
  \frametitle{Hypervisor options}

  \begin{itemize}
  \item Type 1 (bare metal) vs type 2 (run guest OS atop host OS)
    \begin{itemize}
    \item Not always a clear distinction (KVM somewhere between?)
    \item You may have used Type 2 (Parallels, VirtualBox, etc)
    \end{itemize}
  \item Common large-scale choices
    \begin{itemize}
    \item KVM (used by Google cloud)
    \item Xen (used by Amazon cloud)
    \item HyperV (used by Azure)
    \item vmWare (used in many commercial clouds)
    \end{itemize}
  \end{itemize}
\end{frame}


\begin{frame}
  \frametitle{Performance implications: the good}

  VMs perform well for many workloads:
  \begin{itemize}
  \item Hypervisor CPU overheads pretty low (absent sharing)
  \item
    \href{http://h20195.www2.hpe.com/V2/getpdf.aspx/4AA6-2761ENW.pdf}{May
      be within a few percent on LINPACK loads}
  \item
    \href{http://blogs.vmware.com/performance/tag/linpack}{VMWare
      agrees with this}
  \item Virtual memory (mature tech) extending appropriately
  \end{itemize}
\end{frame}


\begin{frame}
  \frametitle{Performance implications: the bad}

  Virtualization does have performance impacts:
  \begin{itemize}
  \item Contention between VMs has nontrivial overheads
  \item Untuned VMs
    \href{http://blog.pierreroudier.net/wp-content/uploads/2015/08/rc25482.pdf}{may
      miss important memory features}
  \item Mismatched scheduling of VMs can slow multi-CPU runs
  \item I/O virtualization is still costly
  \end{itemize}
  Does it make sense to do big PDE solves on VMs yet?  Maybe not, but...
\end{frame}


\begin{frame}
  \frametitle{Performance implications}

  VM performance is a {\em fast} moving target:
  \begin{itemize}
  \item VMs are important for isolation and utilization
    \begin{itemize}
    \item Important for economics of rented infrastructure
    \end{itemize}
  \item Economic importance drives a lot
    \begin{itemize}
    \item Big topic of academic systems research
    \item Lots of industry and open source R\&D (HW and SW)
    \end{itemize}
  \end{itemize}
  Scientific HPC {\em will} ultimately benefit, even if not the driver.
\end{frame}


\begin{frame}
  \frametitle{VM performance punchline}

  \begin{itemize}
  \item VM computing in clouds will not give ``bare metal''
    performance
    \begin{itemize}
    \item If you have
      \href{https://cloudplatform.googleblog.com/2017/10/new-compute-engine-machine-types.html}{96
        vCPUs and 624 GB RAM}, maybe you can afford a couple percent hit?
    \end{itemize}
  \item Try it before you knock it
    \begin{itemize}
    \item Much depends on workload
    \item And remember: performance comparisons are hard!
    \item And the picture will change next year anyhow
    \end{itemize}
  \end{itemize}
\end{frame}


\begin{frame}
  \frametitle{Containers}
\end{frame}


\begin{frame}
  \frametitle{Why virtualize?}

  A not-atypical coding day:
  \begin{enumerate}
  \item Build code (four languages, many libraries)
  \item Doesn't work; install missing library
  \item Requires different version of a dependency
  \item Install new version, breaking different package
  \item Swear, coffee, go to 1
  \end{enumerate}
\end{frame}


\begin{frame}
  \frametitle{Application isolation}

  \begin{itemize}
  \item Desiderata: Codes operate independently on same HW
    \begin{itemize}
    \item Isolated HW: memory spaces, processes, etc (OS handles)
    \item Isolated SW: dependencies, dynamic libs, etc (OS shrugs)
    \end{itemize}
  \item Many tools for isolation
    \begin{itemize}
    \item VM: strong isolation, heavy weight
    \item Python virtualenv: language level, partial isolation
    \item Conda env, modules: still imperfect isolation
    \end{itemize}
  \end{itemize}
\end{frame}


\begin{frame}
  \frametitle{Application portability}

  \begin{itemize}
  \item Desiderata: develop on my laptop, run elsewhere
    \begin{itemize}
    \item Even if ``elsewhere'' refers to a different Linux distro!
    \end{itemize}
  \item What about autoconf, CMake, etc?
    \begin{itemize}
    \item Great at finding {\em some} library that satisfies deps
    \item Maintenance woes: bug on a system I can't reproduce
    \end{itemize}
  \item Solution: Package code and deps in VM?
    \begin{itemize}
    \item But what about performance, image size?
    \end{itemize}
  \end{itemize}
\end{frame}


\begin{frame}
  \frametitle{Containers}

  \begin{itemize}
  \item Instead of virtualizing HW, virtualize OS
  \item Container image includes library deps, config files, etc
  \item Running container has own
    \begin{itemize}
    \item Root filesystem (no sharing libs across containers)
    \item Process space, IPC, TPC sockets
    \end{itemize}
  \item Can run on VM or on bare metal
  \end{itemize}
\end{frame}


\begin{frame}
  \frametitle{Container landscape}

  \begin{itemize}
  \item \href{https://www.docker.com/}{Docker} dominates
  \item \href{https://coreos.com/rkt/}{rkt} is an up-and-coming alternative
  \item Several others (see
    \href{https://coreos.com/rkt/docs/latest/rkt-vs-other-projects.html}{this
      comparison})
  \item Multiple efforts on containers for HPC
    \begin{itemize}
    \item \href{https://www.nersc.gov/research-and-development/user-defined-images/}{Shifter}: Docker-like user-defined images for HPC systems
    \item \href{http://singularity.lbl.gov/}{Singularity}: Competing system
    \end{itemize}
  \end{itemize}
\end{frame}


\begin{frame}
  \frametitle{Containers vs VMs?}

  \begin{itemize}
    \item VMs: Different OS on same HW
      \begin{itemize}
      \item What if I want Windows + Linux on one machine?
      \item Good reason for running VMs locally, too!
      \end{itemize}
    \item VMs: Strong isolation between jobs sharing HW (security)
      \begin{itemize}
      \item OS is supposed to isolate jobs
      \item What about shared OS, one malicious user with root kit?
      \item Hypervisor has smaller attack surface
      \end{itemize}
    \item Containers: one OS, weaker isolation, but lower overhead
  \end{itemize}
\end{frame}


\begin{frame}
  \frametitle{XaaS and the cloud}
\end{frame}

\begin{frame}
  \frametitle{IaaS: Infrastructure}

  \begin{itemize}
  \item Low-level compute for rent
    \begin{itemize}
    \item Computers (VMs or bare metal)
    \item Network (you pay for BW)
    \item Storage (virtual disks, storage buckets, DBs)
    \end{itemize}
  \item Focus of the discussion so far
  \end{itemize}
\end{frame}

\begin{frame}
  \frametitle{PaaS: Platform}

  \begin{itemize}
  \item Programmable environments above raw machines
  \item Example: Wakari and other Python NB hosts
  \end{itemize}
\end{frame}

\begin{frame}
  \frametitle{SaaS: Software}

  \begin{itemize}
  \item Relatively fixed SW package
  \item Example: GMail
  \end{itemize}
\end{frame}

\begin{frame}
  \frametitle{The big three}

  \begin{itemize}
  \item Amazon Web Services (AWS): first mover
  \item Google Cloud Platform: better prices?
  \item Microsoft Azure: only one with Infiniband
  \end{itemize}
\end{frame}

\begin{frame}
  \frametitle{The many others: HPC IaaS}

  \begin{itemize}
  \item \href{https://www.cac.cornell.edu/services/cloudservices.aspx}{RedCloud}: Cornell local
  \item \href{https://www.nimbix.net/}{Nimbix}
  \item \href{http://www.sabalcore.com/}{Sabalcore}
  \item \href{https://pod.penguincomputing.com/hpc_on_demand}{Penguin-on-Demand}
  \end{itemize}
\end{frame}

\begin{frame}
  \frametitle{The many others: HPC PaaS/SaaS}

  \begin{itemize}
  \item \href{http://www.rescale.com/}{Rescale}: Turn-key HPC and simulations
  \item \href{https://pod.penguincomputing.com/}{Penguin On Demand}: Bare-metal IaaS or PaaS
  \item
    \href{https://www.mathworks.com/products/parallel-computing/matlab-parallel-cloud/}{MATLAB
      Cloud}: One-stop shopping for parallel MATLAB cores
  \item \href{https://cyclecomputing.com/}{Cycle computing}: PaaS on clouds (e.g. Google, Amazon, Azure)
  \item \href{https://www.simscale.com/}{SimScale}: Simulation from your browser
  \item \href{https://www.totalcae.com/}{TotalCAE}: Turn-key private or public cloud FEA/CFD
  \item \href{https://www.cpu-24-7.com/}{CPU 24/7}: CAE as a Service
  \end{itemize}
\end{frame}


\end{document}
