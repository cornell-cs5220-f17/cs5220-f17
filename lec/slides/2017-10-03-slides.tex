\documentclass{beamer}

% Should be documentclass beamer

\mode<presentation>
{
%  \usetheme[hideothersubsections]{PaloAlto}
  \usetheme{metropolis}
  \setbeamercovered{transparent}
}

\usepackage{amsfonts}
\usepackage{amsmath}
\usepackage{amssymb}
\usepackage{color}
\usepackage{tikz}
\usepackage{pgfplots}
\usepackage{listings}
\usepackage{courier}
%\usepackage[utf8]{inputenc}
%\usepackage[russian]{babel}

\lstset{
  numbers=left,
  basicstyle=\ttfamily\footnotesize,
  numberstyle=\tiny\color{gray},
  stepnumber=1,
  numbersep=10pt,
}

\newcommand{\iu}{\ensuremath{\mathrm{i}}}
\newcommand{\bbR}{\mathbb{R}}
\newcommand{\bbC}{\mathbb{C}}
\newcommand{\calV}{\mathcal{V}}
\newcommand{\calW}{\mathcal{W}}
\newcommand{\macheps}{\epsilon_{\mathrm{mach}}}
\newcommand{\matlab}{\textsc{Matlab}}

\newcommand{\ddiag}{\operatorname{diag}}
\newcommand{\fl}{\operatorname{fl}}
\newcommand{\nnz}{\operatorname{nnz}}
\newcommand{\tr}{\operatorname{tr}}
\renewcommand{\vec}{\operatorname{vec}}

\newcommand{\vertiii}[1]{{\left\vert\kern-0.25ex\left\vert\kern-0.25ex\left\vert #1
    \right\vert\kern-0.25ex\right\vert\kern-0.25ex\right\vert}}
\newcommand{\ip}[2]{\langle #1, #2 \rangle}
\newcommand{\ipx}[2]{\left\langle #1, #2 \right\rangle}
\newcommand{\order}[1]{O( #1 )}

\newcommand{\kron}{\otimes}


\newcommand{\hdr}[2]{
  \title[CS 5220, Fall 2017]{CS 5220: #2}
  \author{David Bindel}
  \date{#1}
}


\hdr{2017-10-03}{Heterogeneity and accelerators}

\begin{document}

\begin{frame}
  \titlepage
\end{frame}


\begin{frame}
  \frametitle{Reminder: Totient cluster structure}

  Consider:
  \begin{itemize}
  \item Each core has vector parallelism
  \item Each chip has six cores, shares memory with others
  \item Each box has two chips, shares memory
  \item Each box has two \alert{Xeon Phi accelerators}
  \item Eight instructional nodes, communicate via Ethernet
  \end{itemize}
  Common layout (more nodes and better networks at high end)

\end{frame}


\begin{frame}
  \frametitle{Accelerator devices}

  \begin{itemize}
  \item NVidia GPUs
  \item Intel Xeon Phi (aka MIC)
  \item AMD Radeon Pro
  \item Google Tensor Processing Units (TPUs)
  \item Arria (Intel) and Altera FPGAs
  \item Lake Crest, Knights Mill, etc?
  \end{itemize}
\end{frame}


\begin{frame}
  \frametitle{General accelerator scheme}

  \begin{quote}
    If you were plowing a field, which would you rather use: Two
    strong oxen or 1024 chickens? \\
    \hfill --- Seymour Cray
  \end{quote}
  \begin{itemize}
  \item Host computer
    \begin{itemize}
    \item General purpose
    \item Usually multi-core
    \end{itemize}
  \item Accelerator
    \begin{itemize}
    \item Specialized for particular workloads
    \item Often many specialized cores (many-core)
    \item May have a non-x86 ISA, needs different compilers
    \item More ``exotic'' HW support (half precision, wide vecs, etc)
    \item Often has {\em independent memory}
    \end{itemize}
  \end{itemize}
    
\end{frame}


\begin{frame}
  \frametitle{Some historical perspective}

  \begin{itemize}
  \item 1970s -- early 1990s: vector supercomputers
  \item But games pay more than science!
    \begin{itemize}
    \item Mid-late 90s: SIMD vectors in CPUs (for graphics)
    \item Also 90s: Special-purpose GPUs
    \item Early 2000s: Programmable GPUs, rise of GPGPU
    \end{itemize}
  \item And the pendulum swings
    \begin{itemize}
    \item 2007: NVidia Tesla + first version of NVidia CUDA
    \item 2010: Knight's Ferry
    \item 2012-13: Knight's Landing (first commercial Xeon Phi)
    \item Today: mostly NVidia, Intel trailing, AMD a ways back
      \begin{itemize}
      \item NB: Knight's Landing Xeon Phi can operate independently!
      \end{itemize}
    \end{itemize}
  \item More recent accelerators target deep learning
  \end{itemize}
\end{frame}


\begin{frame}
  \frametitle{Accelerator options}

  \begin{itemize}
  \item NVidia GPUs
    \begin{itemize}
    \item Amazon EC2, Google GCE, MS Azure
    \item Summit (ORNL)
    \item Sierra (LLNL)
    \end{itemize}
  \item Intel Xeon Phi
    \begin{itemize}
    \item Totient cluster!
    \item Tianhe-2
    \item TACC Stampede
    \item Aurora (Argonne)
    \end{itemize}
  \end{itemize}
\end{frame}


\begin{frame}
  \frametitle{Same old song...}

  \begin{itemize}
  \item For performance, we need:
    \begin{itemize}
    \item Stern warnings against magical thinking
    \item Enough about HW to reason about performance gotchas
    \item Careful attention to memory issues
    \item Pointers to appropriate programming models
    \end{itemize}
  \item What's different?
    \begin{itemize}
    \item Many more cores/threads
    \item Lots of data parallelism
    \item New? NVidia lore harkens to Cray vector days!
    \end{itemize}
  \end{itemize}
\end{frame}


\begin{frame}
  \frametitle{Programming models}

  \begin{itemize}
  \item \href{https://developer.nvidia.com/gpu-accelerated-libraries}{Call
      a library}!
    \begin{itemize}
    \item This is often the fastest way to faster code
    \item Remember trying to beat BLAS in P1?
    \end{itemize}
  \item CUDA (NVidia only)
  \item OpenCL (clunkier, works with more)
  \item OpenACC
  \item OpenMP
  \item Novel languages (Simit, Julia, ...)
  \end{itemize}
\end{frame}


\begin{frame}
  \frametitle{Totient Phi}

  \href{https://ark.intel.com/products/71992/Intel-Xeon-Phi-Coprocessor-5110P-8GB-1_053-GHz-60-core}{Xeon Phi 5110P}
  \begin{itemize}
  \item Came out late 2012 (now end-of-life)
  \item 60 cores (modified Pentium)
  \item 4 way hyperthreading / 240 hardware threads
  \item AVX512 support (wide vector units)
  \item Base frequency of 1.05 GHz
  \item Ring network on chip
  \item 32K L1 data/instruction cache per core
  \item 30 MB L2 (512K/core) and 8 GB RAM
  \end{itemize}
  Program with OpenMP + directives, OpenCL, Cilk/Cilk+, libraries
\end{frame}


\begin{frame}
  \frametitle{Phi programming}

  \begin{itemize}
  \item Knight's Landing -- maybe just ssh in
  \item Offload mode slides adapted from
    \href{https://portal.tacc.utexas.edu/documents/13601/901837/offload_slides_DJ2013-3.pdf}{TACC
      talk}
  \end{itemize}
\end{frame}

\begin{frame}[fragile]
  \frametitle{Easy perf (Automatic Offloading)}

  Supposing {\tt foo} uses BLAS for performance:
\begin{lstlisting}
  # In Makefile
  icc -qopenmp -mkl foo.c -o foo.x

  # In PBS script
  export MKL_MIC_ENABLE=1
  export OMP_NUM_THREADS=12
  export MIC_OMP_NUM_THREADS=240
  ./foo.x
\end{lstlisting}
  Actually divides work across host and MIC
\end{frame}


\begin{frame}[fragile]
  \frametitle{Compiler-assisted offload: Hello World}

\begin{lstlisting}
  #include <stdio.h>
  #include <omp.h>

  int main()
  {
      int nprocs;

      #pragma offload target(mic)
      nprocs = omp_get_num_procs();    // On MIC

      printf("nprocs = %d\n", nprocs); // On host
      return 0;
  }
\end{lstlisting}
Can have OpenMP on either host or MIC.
\end{frame}


\begin{frame}[fragile]
  \frametitle{Compiler-assisted offload: Hello World}

\begin{lstlisting}
  #include <stdio.h>
  #include <omp.h>

  int main()
  {
      int nprocs;

      #pragma offload target(mic:0)
      nprocs = omp_get_num_procs();    // On MIC 0 (vs MIC 1)

      printf("nprocs = %d\n", nprocs); // On host
      return 0;
  }
\end{lstlisting}
\end{frame}


\begin{frame}
  \frametitle{Compiler-assisted offload}

  Always generate host code, generate code for MIC in
  \begin{itemize}
  \item {\tt offload} regions
  \item Functions marked with {\tt \_\_declspec(target(mic))}
  \end{itemize}
  Can also mark global variables with {\tt \_\_declspec(target(mic))}
\end{frame}


\begin{frame}
  \frametitle{Offload off-stage}

  Execution behind the scenes:
  \begin{itemize}
  \item Detect MICs
  \item Allocate/associate MIC memory
  \item Transfer data to MIC
  \item Execute on MIC
  \item Transfer data from MIC
  \item Deallocate on MIC
  \end{itemize}
  Can control with clauses
\end{frame}


\begin{frame}
  \frametitle{Data transfer}

  \begin{itemize}
  \item {\tt in}, {\tt out}, {\tt inout} clauses: declare how
    variables transfer
  \item {\tt alloc\_if}, {\tt free\_if}: manage allocation for associated dynamic
    arrays on host and accelerator
  \end{itemize}
\end{frame}


\begin{frame}[fragile]
  \frametitle{Compiler-assisted offload}

\begin{lstlisting}
  __declspec(target(mic))
  void something_fancy(int n, double* x) {...}
  
  int main()
  {
      int n = 100;
      double* x = (double*) memalign(64, n*sizeof(double));
      #pragma offload mic \
      inout(x : length(n) alloc_if(1) free_if(1))
      something_fancy(n, x);
      // Do something with x on host
      free(x);
      return 0;
  }
\end{lstlisting}  
\end{frame}


\begin{frame}[fragile]
  \frametitle{Asynchronous execution}

\begin{lstlisting}
  int n = 123;
  #pragma offload target(mic) signal(&n)
  act_very_slowly();
  do_something_on_host();
  #pragma mic offload_wait target(mic) wait(&n)
\end{lstlisting}  
\end{frame}


\begin{frame}
  \frametitle{Desiderata}

  \begin{itemize}
  \item Lots of parallel work
    \begin{itemize}
    \item Vectorized, OpenMP, etc -- both host and MIC
    \end{itemize}
  \item Not too much data transfer
    \begin{itemize}
    \item It's expensive!
    \item Re-use data transfers to MIC if possible
    \end{itemize}
  \end{itemize}
  Writing ``modern'' code tends to be good for both sides...
\end{frame}


\begin{frame}
  \frametitle{What about GPUs?}

  Lots of good references out there:
  \begin{itemize}
  \item
    \href{http://proquest.safaribooksonline.com/book/software-engineering-and-development/9780124159921}{Programming
      Massively Parallel Processors (Kirk and Hwu)} -- available
    online via Cornell library subscription (Safari)
  \item
    \href{http://docs.nvidia.com/cuda/pdf/CUDA_C_Programming_Guide.pdf}{CUDA
      C Programming Guide}
  \item
    \href{http://docs.nvidia.com/cuda/pdf/CUDA_C_Best_Practices_Guide.pdf}{CUDA
      C Best Practices Guide}
  \item
    \href{http://people.maths.ox.ac.uk/gilesm/cuda/}{Oxford CUDA short course}
  \end{itemize}
  Lots of details!  But basic ideas constant: regular computation,
  expose parallelism, exploit locality, minimize memory traffic.
  
\end{frame}


\begin{frame}
  \frametitle{Basic architecture (NVidia GPUs)}

  \begin{itemize}
  \item Array of Streaming Multiprocessors (SMs)
  \item Single Instruction Multiple Thread (SIMT)
    \begin{itemize}
    \item Operate with {\em warp} of 32 threads
    \item Each thread execs same instructions at once
    \item Some may be inactive (for conditional exec)
    \end{itemize}
  \item Exec a warp at a time (want {\em lots} of parallel work!)
  \item Organize threads into logical grids of blocks
  \item Several types of device memory
  \end{itemize}
\end{frame}


\begin{frame}
  \frametitle{How to program?}

  Call a library!
  \begin{itemize}
  \item \href{http://icl.cs.utk.edu/magma/}{MAGMA} for doing NLA
  \item cuBLAS, cuFFT, etc otherwise
  \end{itemize}
  But sometimes you need a little lower level.
\end{frame}


\begin{frame}
  \frametitle{NVidia CUDA}

  Compute Unified Device Architecture.  Three basic ideas:
  \begin{itemize}
  \item Hierarchy of thread groups
  \item Shared memories
  \item Barrier synchronization
  \end{itemize}
  
\end{frame}


\begin{frame}
  \frametitle{Threads and kernels}

  Idea:
  \begin{itemize}
  \item Define {\em kernel} that runs on GPU
  \item Exec kernel on $N$ parallel threads
  \item Different work according to thread index
  \end{itemize}
\end{frame}


\begin{frame}[fragile]
  \frametitle{Hello world}

\begin{lstlisting}
__global__ 
void vecAdd(float* A, float* B, float* C) {
    int i = threadIdx.x;
    C[i] = A[i] + B[i];
}

int main()
{
    // ...
    // Execute with N threads
    vecAdd<<1,N>>(A, B, C);
    // ...
}
\end{lstlisting}
\end{frame}


\begin{frame}
  \frametitle{Hello world}

  \begin{itemize}
  \item Declare {\tt \_\_global\_\_} to run on device
    \begin{itemize}
    \item {\tt \_\_device\_\_} for call/exec on device
    \item {\tt \_\_host\_\_} for all on host (or don't annotate)
    \end{itemize}
  \item Call is {\tt kernel<<nBlk,nThread>>(args)}
  \item Blocks/threads in 1-3D logical index spaces
    \begin{itemize}
    \item Threads form blocks, blocks form grids
    \item IDs are {\tt blockIdx} and {\tt threadIdx} structs
    \item {\tt gridDim} gives blocks/grid
    \item {\tt blockDim} gives threads/block
    \item Each struct has {\tt x}, {\tt y}, {\tt z} fields
    \item Under the hood: 1D space
    \item At most 1024 threads per block
    \end{itemize}
  \end{itemize}
\end{frame}



\begin{frame}[fragile]
  \frametitle{Hello world}

\begin{lstlisting}
__global__ 
void vecAdd(float* A, float* B, float* C, int N) {
    int i = blockIdx.x * blockDim.x + threadIdx.x;
    if (i < N) C[i] = A[i] + B[i];
}

int main()
{
    // ...
    // Execute with N threads
    vecAdd<<1,N>>(A, B, C, N);
    // ...
}
\end{lstlisting}
\end{frame}


\begin{frame}[fragile]
  \frametitle{Where is the data?}

  Explicitly manage device data and transfers:
\begin{lstlisting}
    cudaMalloc((void**)&d_A, size);
    cudaMemcpy(d_A, h_A, size, cudaMemcpyHostToDevice);
    // Do something on device
    cudaMemcpy(h_C, d_C, size, cudaMemcpyDeviceToHost);
    cudaFree(d_A);
\end{lstlisting}
  ... and we have to malloc/free corresponding device data.
\end{frame}


\begin{frame}
  \frametitle{Shared memory and barriers}
  
  Device has several types of memory
  \begin{itemize}
  \item Per-thread: registers, local memory
  \item Per-block: shared memory ({\tt \_\_shared\_\_})
  \item Per-grid: global memory, constant memory
  \end{itemize}
  Synchronize access to shared/global memory with {\tt
    \_\_synchthreads()} (barrier)
\end{frame}


\begin{frame}
  \frametitle{And so forth}

  \begin{itemize}
  \item Other memory types (texture, surface)
  \item Asynchronous execution
  \item Streams and events
  \item ... and the programming guide is 300 pages!
  \end{itemize}
\end{frame}


\begin{frame}
  \frametitle{So now what?}

  So far we have seen
  \begin{itemize}
  \item Two accelerator HW platforms
  \item Two programming models
  \item Same old concerns with lots of new details
  \end{itemize}
  Should be asking
  \begin{itemize}
  \item Is there a better way than low-level mucking about?
  \item What if I want to use this in a larger code?
  \end{itemize}
  Both great questions!  Let's pick them up next time.
  
\end{frame}

% Evolution of compute hardware and the ``zero billion dollar a year''
% Different words, same tune!  And many of the same words.  Basic
%   pattern remains the same: enough about the HW to reason about
%   performance gotchas (with particular reference to memory);
%   issues of how to find parallelism, watch for memory bottlenecks,
%   etc.  Use libraries when possible, and be appropriately skeptical
%   of black magic, unfair comparisons, etc.
% Some history
% HW: NVidia and Intel, lots of others.  Multi-core vs manycore (and
%   1024 chickens)
% Cray and ``everything old is new''
% Programming models: CUDA, OpenCL, OpenACC, OpenMP, C++AMP,
%    libraries, novel languages (e.g. Simit, explicit Julia support)
% Mixed model programming
% Other places to learn?  Pointers to books, slide decks, courses,
%   etc.


\end{document}
